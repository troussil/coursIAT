\documentclass[a4paper,francais]{article}
\usepackage[utf8]{inputenc}
\usepackage[T1]{fontenc}
\usepackage[french]{babel}

\usepackage{subfig}
\usepackage{graphicx}
\graphicspath{{fig/}}

\usepackage{amsmath}
\usepackage{amssymb}
\usepackage{amsthm}
\usepackage{cancel}

\usepackage{hyperref}

\usepackage{cprotect} %verbatim in footnote

\newcommand{\cad}{c.-à-d.}
\newcommand{\Z}{{\ensuremath\mathbb{Z}}}
\newcommand{\N}{{\ensuremath\mathbb{N}}}
\newcommand{\R}{{\ensuremath\mathbb{R}}}
\newtheorem{Theorem}{Theorem}

%-------- enable or disable correction -----------------------------
\theoremstyle{definition}
\newtheorem{exercice}{Exercice}[section]
\newtheorem*{solution}{Solution}

\usepackage{comment}
%\excludecomment{solution}% commenter/décommenter pour afficher/effacer l'impression des solutions


\title{Optimisation linéaire}
\author{Tristan Roussillon}

\begin{document}

\maketitle

Dans ce TD est abordé le problème de l'optimisation linéaire, défini,
pour des fonctions $f$ et $\{g_i\}_{\{1, \dots, m\}}$ linéaires (ou affines),
comme :
\[
\text{(PL)} \left\{
\begin{array}{c}
  \min_x \ f(x) \ \text{tel que :} \\
  g_i(x) \leq 0, \ i \in I = \{1, \dots, m\} \\
  x \in \R^n \\
\end{array}
\right.
\]

(PL) est résolu en pratique par la méthode du \emph{simplexe}
ou celle \emph{des points intérieurs}. La bibliothèque python
\texttt{SciPy}, par exemple, fournit des
\href{https://docs.scipy.org/doc/scipy/reference/generated/scipy.optimize.linprog.html}{solveurs}
basés sur ces deux méthodes.

\section{Formulation matricielle canonique et standard}

La plupart des solveurs attendent la description du problème sous une forme matricielle particulière.

\begin{tabular}{c|c}
forme canonique & forme standard \\ \hline
\begin{minipage}{.4\textwidth}
  \[
\left\{
\begin{array}{c}
  \min_x \ c^T x \ \text{tel que :} \\
  Ax \leq b \\
  x \geq 0 \\
\end{array}
\right.
\]
\end{minipage}
&
\begin{minipage}{.4\textwidth}
\[
\left\{
\begin{array}{c}
  \min_x \ c^T x \ \text{tel que :} \\
  Ax = b \\
  x \geq 0 \\
\end{array}
\right.
\]
\end{minipage}
\end{tabular}

\begin{exercice}
  Expliquez pourquoi on ne perd pas en généralité en supposant (i)
  que la fonction objectif s'écrit comme un produit scalaire, (ii)
  que $x \geq 0$. 
  Expliquez comment passer de la forme canonique à la forme standard.
\end{exercice}

\begin{solution}
  (i) Si la fonction objectif est affine, on peut ajouter une variable,
  contrainte à être égale à 1. 
  
  (ii) S'il existe un variable $x_k$ pouvant prendre n'importe quelle valeur réelle,
  on pourra remplacer $x_k$ par la différence $x_k^+ - x_k^-$ de deux variables
  $x_k^+ - x_k^-$ astreintes, elles, à ne prendre que des valeurs non-négatives.

  Enfin, pour passer de la forme canonique à la forme standard, il convient de remplacer
  toute inégalité par une égalité et une inégalité de signe
  en introduisant une variable d'écart. Par exemple $g(x) \leq 0$ est équivalent
  à $g(x) + y = 0$ et $y \geq 0$.
\end{solution}

\begin{exercice}
  \label{ex:ex}
  Dans (PL), on a $f(x) = -x_1 + x_2 + 1$, $g_1(x) = -x_1 + x_2$, $g_2(x) = -x_2$, 
  et $g_3(x) = x_1 + x_2 - 1$. Donnez une représentation graphique du problème
  et déduisez-en le minimum global. \'Ecrivez le problème dans sa forme standard.
\end{exercice}

\begin{solution}
Le minimum global est $x^\star = (1,0)$, avec $f(x^\star) = 0$ (Fig.~\ref{fig:exPL}).  
\begin{figure}
  \includegraphics[width=0.5\textwidth]{PL}
  \caption{Représentation graphique du problème de l'exercice~\ref{ex:ex}.
    En bleu, les numéros des contraintes}
  \label{fig:exPL}
\end{figure}
  
  \begin{itemize}
  \item Comme je n'ai pas de contrainte de signe sur $x_1$, je le remplace
    par la différence $x_1 - x_3$ de deux variables à valeurs non-négatives.
  \item Pour avoir une fonction objectif linéaire, j'introduis une variable
    $x_4$ astreinte à être égale à $1$.
  \item Pour les contraintes $1$ et $3$, j'introduis deux variables d'écart $x_5$ et $x_6$
    à valeurs non-négatives. 
  \end{itemize}
En résumé, on a 
\[
\left\{
\begin{array}{cl}
  \min_x \ -x_1 + x_2 + x_3 + x_4 \ \text{tel que :} & \\
  x_1, \dots, x_6 \geq 0 & \text{(contrainte 2 et non-négativité)} \\
  x_4 = 1 & \text{(fct. obj. linéaire)} \\
  -x_1 + x_2 + x_3 + x_5  = 0 & \text{(contrainte 1)} \\
  x_1 + x_2 - x_3 + x_6 = 0   & \text{(contrainte 3)} \\
\end{array}
\right.
\]
On a donc la forme standard en définissant
\begin{itemize}
\item $x^T := (x_1, \dots, x_6) \in \R^6$
\item $c^T := (-1, 1, 1, 1, 0, 0)$,
\item $b^T := (1, 0, 0)$
\item $A :=
  \left(
  \begin{array}{cccccc}
    0 & 0 & 0 & 1 & 0 & 0 \\
    -1 & 1 & 1 & 0 & 1 & 0 \\
    1 & 1 & -1 & 0 & 0 & 1 \\
  \end{array}
  \right)$
\end{itemize}
\end{solution}

\section{Condition d'optimalité nécessaires et suffisantes}

Comme la fonction objectif et les contraintes sont linéaires, non seulement
l'hypothèse de qualifications des contraintes est vérifiée, mais les conditions
nécessaires d'optimalité de Kuhn et Tucker sont suffisantes.

Ces dernières s'expriment ainsi : il existe $\{\lambda_i \geq 0\}_{i \in I}$ tels que:
\[
(\text{KT})
\left\{
\begin{array}{l}
  -{\nabla f}(x) = \sum_{i \in I} \lambda_i {\nabla g_i}(x) \\
  \lambda_i g_i(x) = 0, \ \forall i \in I \\
\end{array}
\right.
\]

\begin{exercice}
Vérifiez que le minimum global de l'exercice~\ref{ex:ex} satisfait (KT). 
\end{exercice}

\begin{solution}
  D'après la représentation graphique, $g_2$ et $g_3$ sont saturés, donc
  $\lambda_1 = 0$, $g_2(x) = 0$ et $g_3(x) = 0$. Or, pour tout $x$,
  ${\nabla f}^T(x) = (-1,1)$, ${\nabla g_2}^T(x) = (0,-1)$ et ${\nabla g_2}^T(x) = (1,1)$,
  d'où
\[
\left\{
\begin{array}{l}
  - x_2 = 0 \\
  x_1 + x_2 - 1 = 0 \\
  \lambda_2(0) + \lambda_3 (1) = 1 \\
  \lambda_2(-1) + \lambda_3 (1) = -1 
\end{array}
\right.
\]
ce qui donne
\[
\left\{
\begin{array}{l}
  x_2 = 0 \\
  x_1 = 1 \\
  \lambda_3 = 1 \\
  \lambda_2 = 2 
\end{array}
\right.
\]
\end{solution}

\begin{exercice}
  Exprimez les conditions d'optimalité avec les notations de la forme matricielle standard. 
\end{exercice}

\begin{solution}
  On sépare les coefficients en deux vecteurs :
  $\lambda^= \in \R^m$ (pour les $m$ contraintes d'égalité) et
  $\lambda^\geq \in \R^n$ (pour les $n$ contraintes de positivité).

  Par ailleurs, les gradients sont constants pour tout $x$.
  Celui de $f$ est égal à $c$.
  Si $g_i$ décrit une contrainte d'égalité, son gradient est
  égal à la $i$-ème ligne de $A$ et si $g_i$ décrit une
  contrainte de positivité sur la composante $k$, son gradient
  est un vecteur dont la $k$-ième composante est 1, les autres
  étant nulles. 
  
  Ainsi, $x$ est le minimum global ssi il existe
  $\lambda^= \geq 0$ et $\lambda^\geq \geq 0$ tels que : 
  \[
  \left\{
  \begin{array}{ll}
    -c =  A^T \lambda^= + \lambda^\geq  & \text{(combinaison linéaire)} \\
    Ax = b & \text{(saturation des contraintes d'égalité)} \\
    x^T \lambda^\geq = 0 & \text{(saturation des contraintes de positivité)} 
  \end{array}
  \right.
  \]
  Remarque : la première ligne s'écrit aussi sous la forme d'une inégalité  :
  \[
  A^T \lambda^= \leq -c
  \]
\end{solution}

\section{Transport optimal}

%formulation de Kantorovitch du probleme de TO
%exemple
%mise sous forme standard

%dualite et principe de convoyeur
%https://images.math.cnrs.fr/Transport-optimal-de-mesure-coup-de-neuf-pour-un-tres-vieux-probleme.html?lang=fr
%https://weave.eu/le-transport-optimal-un-couteau-suisse-pour-la-data-science/

%resolution de l'exemple par un solveur quelconque

\end{document}


