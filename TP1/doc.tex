\documentclass[a4paper,francais]{article}
\usepackage[utf8]{inputenc}
\usepackage[T1]{fontenc}
\usepackage[french]{babel}

\usepackage{subfig}
\usepackage{graphicx}
\graphicspath{{fig/}}

\usepackage{amsmath}
\usepackage{amssymb}
\usepackage{amsthm}
\usepackage{cancel}

\usepackage{hyperref}

\usepackage{cprotect} %verbatim in footnote

\newcommand{\cad}{c.-à-d.}
\newcommand{\Z}{{\ensuremath\mathbb{Z}}}
\newcommand{\N}{{\ensuremath\mathbb{N}}}
\newcommand{\R}{{\ensuremath\mathbb{R}}}
\newtheorem{Theorem}{Theorem}

%-------- enable or disable correction -----------------------------
\theoremstyle{definition}
\newtheorem{exercice}{Exercice}[section]
\newtheorem*{solution}{Solution}

\usepackage{comment}
\excludecomment{solution}% commenter/décommenter pour afficher/effacer l'impression des solutions


\title{Optimisation sous contraintes linéaires}
\author{Tristan Roussillon}

\begin{document}

\maketitle

Ce TP aborde le problème de l'optimisation linéaire, défini,
pour des fonctions $f$ et $\{g_i\}_{\{1, \dots, m\}}$ linéaires
\footnote{le terme exact serait \emph{affine}, mais \emph{linéaire}
  est le terme consacré du fait de transformations possibles des
  fonctions affines en fonctions linéaires (cf. section~\ref{sec:formes})}
comme :
\[
\text{(PL)} \left\{
\begin{array}{c}
  \min_{x \in \R^n} \ f(x) \ \text{tel que :} \\
  g_i(x) \leq 0, \ i \in I = \{1, \dots, m\} \\
\end{array}
\right.
\]

(PL) peut
\begin{itemize}
\item être \emph{infaisable} s'il n'y aucune solution,
  c'est-à-dire aucun $x$ ne satisfaisant les contraintes,
\item \emph{ne pas être borné} si, d'une solution donnée, on peut
  toujours se déplacer vers une solution qui fait décroitre $f$,
\item avoir une ou plusieurs solutions optimales, situées
  sur le bord du polytope convexe décrit par les contraintes,
\item être résolu par la méthode du \emph{simplexe}
  (en temps exponentiel au pire cas, mais souvent linéaire en
  pratique) ou celle \emph{des points intérieurs} (en temps
  polynomial et dont les implémentations récentes sont encore
  plus performantes). Il existe une multitude de solveurs,
  mais nous utiliserons la bibliothèque python
  \href{https://docs.scipy.org/doc/}{\texttt{SciPy}} dans
  ce TD.
\end{itemize} 

\section{Conditions d'optimalité nécessaires et suffisantes}
\label{sec:conditions}

\begin{exercice}
  \label{ex:ex}
  Donnez une représentation graphique du problème suivant
  et déduisez-en le minimum global.
  \[
\left\{
\begin{array}{c}
  \min_{x \in \R^2} \ f(x) := -x_1 + x_2 + 1 \ \text{tel que :} \\
  \begin{array}{ll}
    g_1(x) &:= -x_1 + x_2 \leq 0 \\
    g_2(x) &:= -x_2 \leq 0 \\
    g_3(x) &:= x_1 + x_2 - 1 \leq 0 \\
  \end{array} 
\end{array}
\right.
\]
\end{exercice}

\begin{solution}
Le minimum global est $x^\star = (1,0)$, avec $f(x^\star) = 0$ (Fig.~\ref{fig:exPL}).  
\begin{figure}
  \centering
  \includegraphics[width=0.4\textwidth]{PL}
  \caption{Représentation graphique du problème de l'exercice~\ref{ex:ex}.
    En bleu, les numéros des contraintes}
  \label{fig:exPL}
\end{figure}
\end{solution}

Comme la fonction objectif et les contraintes sont affines, non seulement
l'hypothèse de qualifications des contraintes est vérifiée, mais les conditions
nécessaires d'optimalité de Kuhn et Tucker sont suffisantes : 
il existe $\{\lambda_i \ | \ i \in I, \lambda_i \geq 0\}$ et un minimum global
$x^\star$ tels que:
\[
(\text{KT})
\left\{
\begin{array}{ll}
  -{\nabla f}(x^\star) = \sum_{i \in I} \lambda_i {\nabla g_i}(x^\star) & \text{(combinaison linéaire)}\\
  \lambda_i g_i(x^\star) = 0, \ \forall i \in I  & \text{(saturation)}\\
\end{array}
\right.
\]

\begin{exercice}
Vérifiez que le minimum global de l'exercice~\ref{ex:ex} satisfait (KT). 
\end{exercice}

\begin{solution}
  D'après la représentation graphique, $g_2$ et $g_3$ sont saturés, donc
  $g_2(x^\star) = 0$ et $g_3(x^\star) = 0$ par définition et $\lambda_1 = 0$,
  d'après la seconde ligne de (KT). Or, pour tout $x$,
  \begin{itemize}
  \item ${{\nabla f}(x)}^T = (-1,1)$,
  \item ${{\nabla g_2}(x)}^T = (0,-1)$,
  \item ${{\nabla g_3}(x)}^T = (1,1)$,
  \end{itemize}
  d'où le système linéaire de 4 équations et 4 inconnues : 
\[
\left\{
\begin{array}{ll}
  - x^\star_2 &= 0 \\
  x^\star_1 + x^\star_2 - 1 &= 0 \\
  \lambda_2 \left(\begin{array}{c} 0 \\ -1 \end{array}\right)
  + \lambda_3 \left(\begin{array}{c} 1 \\ 1 \end{array}\right)
  &= \left(\begin{array}{c} 1 \\ -1 \end{array}\right)
\end{array}
\right.
\]
qui donne
\[
\left\{
\begin{array}{l}
  x^\star_2 = 0 \\
  x^\star_1 = 1 \\
  \lambda_3 = 1 \\
  \lambda_2 = 2 
\end{array}
\right.
\]
\end{solution}

\section{Forme matricielle}
\label{sec:formes}

La plupart des solveurs attendent la description du problème sous une forme matricielle :

\begin{tabular}{c|c}
forme canonique & forme standard \\ \hline
\begin{minipage}{.4\textwidth}
  \[
\left\{
\begin{array}{c}
  \min_x \ c^T x \ \text{tel que :} \\
  Ax \leq b \\
  x \geq 0 \\
\end{array}
\right.
\]
\end{minipage}
&
\begin{minipage}{.4\textwidth}
\[
\left\{
\begin{array}{c}
  \min_x \ c^T x \ \text{tel que :} \\
  Ax = b \\
  x \geq 0 \\
\end{array}
\right.
\]
\end{minipage}
\end{tabular}

\begin{itemize}
\item $x$ est le vecteur solution du problème. S'il y a $n$ inconnues, c'est un vecteur $n \times 1$.
\item $c$ est un vecteur de même dimension que $x$ et sert à définir la fonction objectif.
\item $A$ et $b$ servent à décrire $m$ inégalités linéaires : $b$ est un vecteur $m \times 1$,
  $A$ est une matrice $m \times n$ et le signe $\leq$ s'applique composante par composante et pour
  toutes les composantes des vecteurs.
\item Enfin, $0$ représente le vecteur nul de dimension $n \times 1$. 
\end{itemize}

\begin{exercice}
  \label{ex:ex-mat}
  Expliquez pourquoi on ne perd pas en généralité en supposant (i)
  que la fonction objectif s'écrit comme un produit scalaire, (ii)
  que $x \geq 0$.
  Exprimez le problème de l'exercice~\ref{ex:ex} dans sa forme
  canonique, puis dans sa forme standard, en expliquant comment
  on peut toujours passer de l'un à l'autre. 
\end{exercice}

\begin{solution}
  (i) Si la fonction objectif est affine (il y a une constante $c_0$ ajoutée à une
  combinaison linéaire des variables), on peut soit l'ignorer, car $x^\star$ qui
  minimise $f'(x^\star)  := f(x^\star) - c_0$, minimise aussi $f(x^\star)$, soit
  ajouter une variable, contrainte à être égale à 1 et destinée à être multipliée
  par $c_0$ de façon à n'avoir plus qu'une combinaison linéaire de variables. 
  
  (ii) S'il existe une variable $x_k$ pouvant prendre n'importe quelle valeur réelle,
  on pourra remplacer $x_k$ par la différence $x'_k - x_k''$ de deux variables
  $x'_k$ et $x_k''$ astreintes, elles, à ne prendre que des valeurs non-négatives.

  Ainsi, pour obtenir la forme canonique de l'exercice~\ref{ex:ex}:
  \begin{itemize}
  \item On a une contrainte de signe sur $x_2$. On peut donc poser $x'_2 := x_2$
    avec $x'_2 \geq 0$ à cause de $g_2$. Pour ajouter une contrainte de signe
    sur $x_1$ (puisqu'il n'y en a pas \emph{a priori}), on pose $x_1 := x'_1 - x'_3$,
    avec $x'_1, x'_3 \geq 0$. 
  \item Pour avoir une fonction objectif linéaire, on va minimiser simplement
    $f'(x'_1, x'_2, x'_3) := -x'_1 + x'_2 + x'_3$. 
  \end{itemize}
  Sous forme matricielle on a donc: 
  \begin{itemize}
  \item $x^T := (x'_1, \dots, x'_3) \in \R^3$
  \item $c^T := (-1, 1, 1)$,
  \item $b^T := (0, 1)$
  \item $A :=
    \left(
    \begin{array}{cccccc}
      -1 & 1 & 1 \\
      1 & 1 & -1 \\
    \end{array}
    \right)$
  \end{itemize}
  Note: les deux lignes de $A$ correspondent aux contraintes $g_1$ et $g_3$, $g_2$
  étant prise en compte dans les contraintes de signe. 
  
  Enfin, pour passer de la forme canonique à la forme standard, il convient de remplacer
  toute inégalité par une égalité et une inégalité de signe
  en introduisant une \emph{variable d'écart}.
  Par exemple $g(x) \leq 0$ est équivalent
  à $g(x) + y = 0$ et $y \geq 0$.

  Ainsi, pour les deux inégalités impliquant les contraintes $g_1$ et $g_3$,
  on ajoute les variables d'écart $x'_4$ et $x'_5$ respectivement pour avoir: 
  \[
  \left\{
  \begin{array}{cl}
    \min \ -x'_1 + x'_2 + x'_3 \ \text{tel que :} & \\
    x'_1, \dots, x'_5 \geq 0 & \text{(contrainte 2 et non-négativité)} \\
    -x'_1 + x'_2 + x'_3 + x'_4  = 0 & \text{(contrainte 1)} \\
    x'_1 + x'_2 - x'_3 + x'_5 = 1   & \text{(contrainte 3)} \\
  \end{array}
  \right.
  \]
  D'où la forme standard en définissant
  \begin{itemize}
  \item $x^T := (x'_1, \dots, x'_5) \in \R^5$
  \item $c^T := (-1, 1, 1, 0, 0)$,
  \item $b^T := (0, 1)$
  \item $A :=
    \left(
    \begin{array}{cccccc}
      -1 & 1 & 1 & 1 & 0 \\
      1 & 1 & -1 & 0 & 1 \\
    \end{array}
    \right)$
  \end{itemize}
\end{solution}

\section{Classification linéaire}
\label{sec:classif}

On suppose donnés des points de $\R^2$, $\{u^j \ | \ {j \in J} \}$ partitionnés en deux sous-ensembles :
un sous-ensemble d'observations positives $S^+ := \{ u^j, j \in J^+ \}$ et un sous-emsemble
d'observations négatives $S^- := \{ u^j, j \in J^- \}$
(avec $J = J^+ \cup J^-$ et $J^+ \cap J^- = \emptyset$).
On veut déterminer une règle de décision linéaire permettant de discriminer au mieux
les observations positives et les observations négatives. Ce problème peut être vu comme la recherche
d'un élément $x \in \R^2$ tel que, en considérant le produit scalaire entre $x$ et les points de $S^+$ et $S^-$,
on ait deux intervalles séparables, aussi écartés que possible l'un de l'autre (Fig.~\ref{fig:classif}).

\begin{figure}[htbp]
  \centering
  \includegraphics[width=0.4\textwidth]{classif}
  \caption{Illustration du problème 2d de la classification linéaire.
    La flèche représente une solution.
    Les intervalles, ici disjoints, des projections des observations
    positives (noir) et négatives (blanc) sont représentés en rouge.  }
  \label{fig:classif}
\end{figure}

\begin{exercice}
  En introduisant deux variables supplémentaires
  $z^-$ et $z^+$ qui, à l'optimal, devraient être égales
  à $\max_{j \in J^-} (x^T u^j)$ et $\min_{j \in J^+} (x^T u^j)$
  (avec $z^+ > z^-$ si les intervalles sont disjoints),
  décrivez le problème décrit ci-dessus comme un problème d'optimisation linéaire.
  Pourquoi est-il nécessaire d'introduire une condition de normalisation sur $x$ ?
\end{exercice}

\begin{solution}
  Le problème s'écrit naturellement ainsi : 
  \[
  \left\{
  \begin{array}{ll}
    \max z^+ - z^- \ \text{tel que} & \\
    x^T u^j \geq z^+ & \forall j \in J^+ \\ 
    x^T u^j \leq z^- & \forall j \in J^- \\
  \end{array}
  \right.
  \]

  Le produit scalaire entre $x$ et les points de $S^+$ et $S^-$ a toujours $\|x\|$
  comme facteur. Par conséquent, plus $x$ est grand et plus les produits scalaires
  seront distants les uns des autres et maximiser l'écart est un problème non
  borné. Il faut ajouter des contraintes de normalisation pour empêcher qu'une
  solution candidate $x$ puisse être arbitrairement grande. 
\end{solution}

\begin{exercice}
  Nous allons maintenant ajouter des contraintes au problème d'optimisation
  de la question précédente afin que l'optimum global vérifie : 
  \begin{enumerate}
  \item\label{Linf} $\|x^\star\|_\infty = 1$,
  \item\label{L1} $\|x^\star\|_1 = 1$.
  \end{enumerate}
  
  Expliquez pourquoi contraindre que chaque composante $x_k, \forall k \in \{1, 2\}$
  soit entre -1 et 1 abouti au cas~\ref{Linf}.
  
  Pour le cas suivant, $\forall k \in \{1, 2\}$, on introduit deux variables $x_k'$ et
  $x_k''$ non-négatives et telles que $x_k := x_k' - x_k''$. Expliquez pourquoi la
  contrainte $\sum_k x_k' + x_k'' \leq 1$ abouti au cas~\ref{L1}. 
\end{exercice}

\begin{solution}
  Dans le cas~\ref{Linf}, la norme $L_\infty$ de $x$ est fixée à $1$.
  Mais au lieu d'ajouter explicitement la contrainte $\max_k |x_k| = 1$
  (qui implique max et valeurs absolues),
  on demande seulement que pour tout $\forall k \in \{1, 2\}$, on ait
  $-1 \leq x_k \leq 1 \Leftrightarrow 0 \leq |x_k| \leq 1$ et pour la même
  raison que celle indiquée dans la réponse à la question précédente
  la norme $L_\infty$ de $x$ sera maximisée pour atteindre $1$ à l'optimal.
  
  Finalement, on a le problème d'optimisation linéaire suivant :  
  \[
  \left\{
  \begin{array}{ll}
    \max z^+ - z^- \ \text{tel que} & \\
    x^T u^j \geq z^+ & \forall j \in J^+ \\ 
    x^T u^j \leq z^- & \forall j \in J^- \\
    -1 \leq x_k \leq 1 & \forall k \in \{1, 2\} \\
  \end{array}
  \right.
  \]

  Dans le cas~\ref{L1}, au lieu d'énoncer explicitement la contrainte
  que la norme $L_1$ vaut $1$, ce qui implique des valeurs absolues, l'astuce consiste
  à réexprimer chaque variable $x_k$ comme la différence de deux variables non-négatives.
  Il suffit alors de vérifier que :
  \[ \sum_k |x_k| \leq \sum_k x_k' + x_k'' \leq 1. \]
  En effet, comme $x_k'$ et $x_k''$ sont non-négatives, $\forall k \in \{1, 2\}$,
  si $x_k \leq 0$, alors $x_k \leq 0 \leq x_k' + x_k''$
  et si $x_k > 0$, alors $x_k = x_k' - x_k'' \leq x_k' + x_k''$,
  d'où $|x_k| \leq x_k' + x_k''$.
  Et comme précédemment, la norme $L_1$ de $x$ sera maximisée pour atteindre $1$ à l'optimal.
  
  Finalement, on a : 
  \[
  \left\{
  \begin{array}{ll}
    \max z^+ - z^- \ \text{tel que} & \\
    \sum_k (x_k' - x_k'') u_k^j \geq z^+ & \forall j \in J^+ \\ 
    \sum_k (x_k' - x_k'') u_k^j \leq z^- & \forall j \in J^- \\
    \sum_k x_k' + x_k'' \leq 1 \\
    x_k' \geq 0 & \forall k \in \{1, 2\} \\
    x_k'' \geq 0 & \forall k \in \{1, 2\} \\
  \end{array}
  \right.
  \]
  
  Remarque : il serait naturel d'opter pour la norme $L_2$, mais cela ferait intervenir
  une contrainte d'égalité quadratique : $\sum_k x_k^2 = 1$. Nous n'aurions plus dans
  ce cas un problème d'optimisation linéaire. 
\end{solution}

\begin{exercice}
  En reprenant le code Python disponible sur Moodle\footnote{
    le code utilise la version 1.3.3 de SciPy qui est la version
    disponible dans les dépôts sur Ubuntu 20.04.}, programmez la résolution du problème
  de classification linéaire correspondant aux deux conditions de normalisation évoquées
  ci-dessus.
  Que vous utilisiez 
  \href{https://docs.scipy.org/doc/scipy-1.3.3/reference/generated/scipy.optimize.linprog.html#scipy.optimize.linprog}{\texttt{linprog}} ou
  \href{https://docs.scipy.org/doc/scipy-1.3.3/reference/generated/scipy.optimize.minimize.html}{\texttt{minimize}},
  le plus délicat est de décrire correctement les données nécessaires au solveur.
  Vous aurez aussi besoin de
  \href{https://docs.scipy.org/doc/numpy/reference/routines.array-creation.html}{créer} et
  \href{https://docs.scipy.org/doc/numpy/reference/routines.array-manipulation.html}{manipuler} des
  tableaux
  \href{https://numpy.org/doc/stable/contents.html#numpy-docs-mainpage}{\texttt{NumPy}}. 
\end{exercice}

\begin{solution}
  Il faut arriver à construire correctement les vecteurs et les matrices nécessaires
  au solveur, comme sur la Fig.~\ref{fig:stdForm} pour le cas~\ref{L1}. Le code complet
  se trouve sur Moodle.  
  
  \begin{figure}[htbp]
  \centering
  \includegraphics[width=0.49\textwidth,page=1]{standardForm}
  \includegraphics[width=0.49\textwidth,page=2]{standardForm}
  \caption{}
  \label{fig:stdForm}
  \end{figure}
\end{solution}

\section{Séparateurs à vaste marge}
\label{sec:svm}

Notez que le problème précédent est exactement celui considéré dans les
méthodes d'apprentissage connues sous le nom de \emph{Support Vector Machines}
\href{https://en.wikipedia.org/wiki/Support-vector_machine}{(SVM)} ou
\emph{Séparateurs à Vaste Marge} en français. Ce sont des classifieurs
linéaires\footnote{Ils peuvent aussi être adaptés au cas où les données
  ne sont pas parfaitement séparables ou au cas où les données sont
  non-linéairement séparables.} 
qui ont de meilleurs capacités de généralisation que le perceptron 
parce qu'ils maximisent la marge de séparation.

Si on note $z := \frac{\min_{j \in J^+} (x^T u^j) + \max_{j \in J^-} (x^T u^j)}{2}$,
la normalisation choisie dans la formulation du SVM consiste à faire
en sorte que $\min_{j \in J^+} (x^T u^j) = z + 1$ et
$\max_{j \in J^-} (x^T u^j) = z - 1$.
Dans ce cas, l'épaisseur euclidienne de la marge est égale à $2/\|x\|_2$.
Mais au lieu de maximiser cette quantité qui fait intervenir la fonction inverse,
de manière équivalente, on minimise, la quantité $\|x\|_2^2$\footnote{
  le carré permet de se débarrasser de la fonction racine carré et
  les facteurs ne changent rien au problème.}.

Les contraintes restent linéaires, mais la fonction objectif est maintenant
quadratique ; il s'agit d'un problème d'optimisation quadratique sous contraintes
linéaires : 
\[
\left\{
\begin{array}{ll}
  \min_{x,z} \|x\|_2^2 \ \text{tel que} & \\
  x^T u^j \geq z + 1 & \forall j \in J^+ \\ 
  x^T u^j \leq z - 1 & \forall j \in J^- \\
\end{array}
\right.
\]

\begin{exercice}
  Complétez le code de la section précédente pour ajouter la résolution du problème
  ci-dessus par l'algorithme de
  \href{https://en.wikipedia.org/wiki/Frank%E2%80%93Wolfe_algorithm}{Franck et Wolfe}. 
\end{exercice}

\end{document}


