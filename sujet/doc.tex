\documentclass[a4paper,francais]{article}
\usepackage[utf8]{inputenc}
\usepackage[T1]{fontenc}
\usepackage[french]{babel}

\usepackage{subfig}
\usepackage{graphicx}
\graphicspath{{fig/}}

\usepackage{amsmath}
\usepackage{amssymb}
\usepackage{amsthm}
\usepackage{cancel}
\usepackage{enumitem}

\usepackage{hyperref}

\usepackage{cprotect} %verbatim in footnote

\newcommand{\cad}{c.-à-d.}
\newcommand{\Z}{{\ensuremath\mathbb{Z}}}
\newcommand{\N}{{\ensuremath\mathbb{N}}}
\newcommand{\R}{{\ensuremath\mathbb{R}}}
\newtheorem{Theorem}{Theorem}
\newtheorem{Exemple}{Exemple}

%-------- enable or disable correction -----------------------------
\theoremstyle{definition}
\newtheorem{exercice}{Exercice}[section]
\newtheorem*{solution}{Solution}

\usepackage{comment}
\excludecomment{solution}% commenter/décommenter pour afficher/effacer l'impression des solutions


\title{Changez de CAP et prenez des couleurs, \\
  Méthodes d'optimisation combinatoire appliquées au Channel Assignment Problem}
\author{Tristan Roussillon}

\begin{document}

\maketitle


\section{Définition du problème}
\label{sec:CAP}

Le problème d'allocation (ou assignation ou attribution) de fréquences consiste
à répartir un ensemble de fréquences sur les stations de base d'un réseau
tout en minimisant le nombre de fréquences utilisées. Il est habituellement
appelé \emph{Channel assignment problem} (CAP) ou encore \emph{minimum span frequency
  assignment problem} (MSFAP) en anglais, mais possède plusieurs variantes.
Le but de cette section est de poser une version de ce problème de manière rigoureuse
\footnote{Vous avez dans ce \href{http://www-sop.inria.fr/members/Frederic.Havet/Cours/channel.pdf}{document}
  dont nous reprenons plusieurs éléments, une description un peu plus approfondie
  des aspects de télécommunication mis en jeu.}.

Nous considérons un ensemble de noeuds $V$. Ils représentent des émetteurs.
Nous avons une fonction $p : V \rightarrow \N$ qui associe à chaque noeud, un nombre
entier appelé \emph{demande}. C'est le nombre de fréquences distinctes souhaité pour
un émetteur. Il y a un ordre total sur les fréquences de sorte qu'elles peuvent être
représentées par un ensemble d'entiers consécutifs $\{1, \dots, t\}$.
Nous avons de plus une fonction $l : V \times V \rightarrow \N$
qui associe à chaque paire de noeuds, un nombre entier qu'on peut appelé \emph{poids}
ou \emph{longueur}. 
Cette fonction décrit les contraintes d'interférence entre les émetteurs :
les fréquences utilisées par deux émetteurs $v_1$ et $v_2$ sont contraintes
d'être distantes de $l(v_1,v_2)$ ou plus afin d'éviter toute interférence.
Remarquons enfin que $V$ et $l$ définissent implicitement un graphe $G(V,E)$, 
où $E$ est défini comme l'ensemble des paires de poids non nul
$\{ (v_1,v_2) \in V \times V \ | \ l(v_1,v_2) > 0 \}$.  
 
Une \emph{solution} est une fonction $\phi : V \rightarrow \mathcal{P}(\{1,\dots,t\})$ telle que
le nombre d'éléments de $\phi(v)$ est égal à $p(v)$ pour chaque $v \in V$\footnote{La
  notation $\mathcal{P}$ désigne l'ensemble des parties d'un ensemble. Pour
  chaque $v \in V$, $\phi(v)$ est donc un sous-ensemble des fréquences $\{1,\dots,t\}$}.
Une solution est \emph{faisable} si pour toute paire $(v_1,v_2) \in V \times V$,
tout $f_1 \in \phi(v_1)$ et tout $f_2 \in \phi(v_2)$, on a $|f_1 - f_2| \geq l(v_1,v_2)$.
L'intervalle (\emph{span}) du problème, noté $\text{span}(V, p, l)$, est le plus petit $t$
tel qu'il existe une solution faisable.
\'Etant donnés $V, p, l$, le problème (CAP) consiste à déterminer $\text{span}(V, p, l)$
ainsi qu'une solution faisable $\phi$.

\begin{Exemple}
  \label{ex:triangle}
  $G$ est un cycle de 3 noeuds, qui ont chacun une demande de $1$.
  Les 3 arêtes sont de poids $3$. Le span est $7$.
\end{Exemple}

\begin{Exemple}
  \label{ex:carre}
  $G$ est un cycle de 4 noeuds, qui ont chacun une demande de $1$.
  Les 4 arêtes sont de poids $3$. Le span est $4$.
\end{Exemple}

\begin{Exemple}
  \label{ex:demande2}
  $G$ est un cycle de 5 noeuds avec en plus une boucle sur chaque noeud.
  Chaque noeud a une demande de $2$. Les boucles ont un poids de $2$,
  tandis que toutes les autres arêtes ont un poids de $1$. Le span est $5$. 
\end{Exemple}

\begin{exercice}
  Expliquez comment on peut transformer tout (CAP) faisant intervenir
  des demandes supérieures à $1$ en un (CAP) dont la demande est $1$
  pour chaque noeud. Illustrez avec l'exemple~\ref{ex:demande2}.
\end{exercice}

Grâce à cette transformation, nous pouvons maintenant considérer sans
perdre en généralité que la demande est $1$ pour chaque noeud.

\begin{exercice}
  La coloration de graphe consiste à trouver le plus petit nombre de
  couleurs permettant d'attribuer une couleur à chacun des noeuds
  d'un graphe, tout en garantissant que deux noeuds reliés par une
  arête sont de couleur différente.
  Expliquez sous quelles conditions (CAP) correspond à un problème
  de coloration de graphe. Illustrez avec l'exemple~\ref{ex:carre}. 
\end{exercice}

La coloration de graphe est un problème NP-complet. Or, (CAP) étant
au moins aussi difficile que la coloration de graphe d'après la
question précédente, on conclut que (CAP) est aussi un problème
NP-complet. 

\section{Résolution du problème}
\label{sec:resolution}

Même si (CAP) est NP-comlet, on peut résoudre certaines instances
de ce problème. Vous avez par exemple pu résoudre des instances de (CAP)
proposées dans les exemples \ref{ex:triangle}, \ref{ex:carre} et
\ref{ex:demande2}. 

\begin{exercice}
  Implémentez\footnote{
    les langages acceptés sont Python, Java, C, C++. } un
  sous-ensemble de ces algorithmes (par ordre de difficulté et points
  dans la note finale) afin de résoudre une instance de (CAP):
  \begin{enumerate}
  \item \emph{greedy} (glouton) : on calcule 
    progressivement une solution en ne considérant qu'une
    nouvelle variable à la fois et sans remettre en cause
    les affectations précédentes.
    Bien sûr, il n'y a aucune garantie d'obtenir à la fin une
    solution optimale, mais c'est un début quand on s'attaque à
    un nouveau problème. 
  \item \emph{backtracking} (retour sur trace) : à chaque fois
    qu'on choisit une nouvelle variable, et pour chaque affectation
    possible de cette variable, on teste récursivement si une
    solution faisable peut être construite à partir de cette
    affectation partielle. Si aucune solution n'est trouvée,
    on revient sur les affectations qui ont été faites précédemment
    (d'où le nom de retour sur trace).
  \item \emph{branch-and-bound} (séparation-évaluation) :
    à chaque fois qu'on choisit une nouvelle variable, et pour
    chaque affectation possible de cette variable (séparation),
    on détermine un minorant de la fonction objectif à partir
    de cette affectation partielle (évaluation), afin de ne tester
    que les cas où le minorant est suffisamment petit par rapport
    à ce qui a déjà été calculé. Le plus délicat est de définir
    la fonction d'évaluation, mais on a la garantie d'avoir à
    la fin l'optimum exact sans avoir tout énuméré. 
  \end{enumerate}
\end{exercice}

\begin{exercice}
  Donnez le résultat de l'exécution de vos programmes (temps et solution)
  pour les exemples jouets \ref{ex:triangle}, \ref{ex:carre} et \ref{ex:demande2},
  ainsi que pour les 9 instances du jeu de données
  \href{http://fap.zib.de/problems/Philadelphia/}{Philadelphia}
  décrites également en annexe. 
\end{exercice}

Vous devez rendre une archive contenant vos programmes et un
court document répondant aux questions du sujet et expliquant
vos choix d'implémentation. 

\appendix

\section{Jeu de données
\href{http://fap.zib.de/problems/Philadelphia/}{Philadelphia}}

Considérons un ensemble de 21 noeuds $V = \{v_1, \dots, v_{21} \}$.

Considérons les demandes décrites par les listes suivantes
(le i-ème élément est la demande pour le i-ème noeud) :
\begin{enumerate}[label=(L\arabic*)]
\item $(8,25,8,8,8,15,18,52,77,28,13,15,31,15,36,57,28,8,10,13,8)$
\item $(5,5,5,8,12,25,30,25,30,40,40,45,20,30,25,15,15,30,20,20,25)$
\item $(20,20,20,20,20,20,20,20,20,20,20,20,20,20,20,20,20,20,20,20,20)$
\item $16,50,16,16,16,30,36,104,154,56,26,30,62,30,72,114,56,16,20,26,16)$
\item $(32,100,32,32,32,60,72,208,308,112,52,60,124,60,144,228,112,32,40,52,32)$
\end{enumerate}

Considérons les longueurs décrites par les matrices suivantes
(à l'intersection de la i-ème ligne et j-ème colonne, on a la
longueur de l'arête reliant $v_i$ et $v_j$, 0 indiquant l'absence
d'arête) :
\begin{enumerate}[label=(M\arabic*)]
\item
\begin{verbatim}
5 2 1 1 0 1 5 5 1 1 0 0 1 1 1 1 1 0 1 1 0 
2 5 2 1 1 1 1 5 5 1 1 0 0 1 1 1 1 1 1 1 1 
1 2 5 2 1 0 1 1 5 5 1 1 0 0 1 1 1 1 1 1 1 
1 1 2 5 2 0 0 1 1 5 5 1 0 0 0 1 1 1 0 1 1 
0 1 1 2 5 0 0 0 1 1 5 5 0 0 0 0 1 1 0 0 1 
1 1 0 0 0 5 2 1 1 0 0 0 5 5 1 1 0 0 1 0 0 
5 1 1 0 0 2 5 2 1 1 0 0 1 5 5 1 1 0 1 1 0 
5 5 1 1 0 1 2 5 2 1 1 0 1 1 5 5 1 1 1 1 1 
1 5 5 1 1 1 1 2 5 2 1 1 0 1 1 5 5 1 1 1 1 
1 1 5 5 1 0 1 1 2 5 2 1 0 0 1 1 5 5 1 1 1 
0 1 1 5 5 0 0 1 1 2 5 2 0 0 0 1 1 5 0 1 1 
0 0 1 1 5 0 0 0 1 1 2 5 0 0 0 0 1 1 0 0 1 
1 0 0 0 0 5 1 1 0 0 0 0 5 2 1 1 0 0 1 0 0 
1 1 0 0 0 5 5 1 1 0 0 0 2 5 2 1 1 0 1 1 0 
1 1 1 0 0 1 5 5 1 1 0 0 1 2 5 2 1 1 2 1 1 
1 1 1 1 0 1 1 5 5 1 1 0 1 1 2 5 2 1 2 2 1 
1 1 1 1 1 0 1 1 5 5 1 1 0 1 1 2 5 2 1 2 2 
0 1 1 1 1 0 0 1 1 5 5 1 0 0 1 1 2 5 1 1 2 
1 1 1 0 0 1 1 1 1 1 0 0 1 1 2 2 1 1 5 2 1 
1 1 1 1 0 0 1 1 1 1 1 0 0 1 1 2 2 1 2 5 2 
0 1 1 1 1 0 0 1 1 1 1 1 0 0 1 1 2 2 1 2 5
\end{verbatim}
\item
\begin{verbatim}
5 2 1 0 0 1 5 5 1 0 0 0 0 1 1 1 0 0 0 0 0 
2 5 2 1 0 0 1 5 5 1 0 0 0 0 1 1 1 0 0 0 0 
1 2 5 2 1 0 0 1 5 5 1 0 0 0 0 1 1 1 0 0 0 
0 1 2 5 2 0 0 0 1 5 5 1 0 0 0 0 1 1 0 0 0 
0 0 1 2 5 0 0 0 0 1 5 5 0 0 0 0 0 1 0 0 0 
1 0 0 0 0 5 2 1 0 0 0 0 5 5 1 0 0 0 0 0 0 
5 1 0 0 0 2 5 2 1 0 0 0 1 5 5 1 0 0 1 0 0 
5 5 1 0 0 1 2 5 2 1 0 0 0 1 5 5 1 0 1 1 0 
1 5 5 1 0 0 1 2 5 2 1 0 0 0 1 5 5 1 1 1 1 
0 1 5 5 1 0 0 1 2 5 2 1 0 0 0 1 5 5 0 1 1 
0 0 1 5 5 0 0 0 1 2 5 2 0 0 0 0 1 5 0 0 1 
0 0 0 1 5 0 0 0 0 1 2 5 0 0 0 0 0 1 0 0 0 
0 0 0 0 0 5 1 0 0 0 0 0 5 2 1 0 0 0 0 0 0 
1 0 0 0 0 5 5 1 0 0 0 0 2 5 2 1 0 0 1 0 0 
1 1 0 0 0 1 5 5 1 0 0 0 1 2 5 2 1 0 2 1 0 
1 1 1 0 0 0 1 5 5 1 0 0 0 1 2 5 2 1 2 2 1 
0 1 1 1 0 0 0 1 5 5 1 0 0 0 1 2 5 2 1 2 2 
0 0 1 1 1 0 0 0 1 5 5 1 0 0 0 1 2 5 0 1 2 
0 0 0 0 0 0 1 1 1 0 0 0 0 1 2 2 1 0 5 2 1 
0 0 0 0 0 0 0 1 1 1 0 0 0 0 1 2 2 1 2 5 2 
0 0 0 0 0 0 0 0 1 1 1 0 0 0 0 1 2 2 1 2 5
\end{verbatim}
\item
\begin{verbatim}
5 2 1 1 0 2 5 5 2 1 0 0 1 2 2 2 1 0 1 1 0 
2 5 2 1 1 1 2 5 5 2 1 0 0 1 2 2 2 1 1 1 1 
1 2 5 2 1 0 1 2 5 5 2 1 0 0 1 2 2 2 1 1 1 
1 1 2 5 2 0 0 1 2 5 5 2 0 0 0 1 2 2 0 1 1 
0 1 1 2 5 0 0 0 1 2 5 5 0 0 0 0 1 2 0 0 1 
2 1 0 0 0 5 2 1 1 0 0 0 5 5 2 1 0 0 1 0 0 
5 2 1 0 0 2 5 2 1 1 0 0 2 5 5 2 1 0 1 1 0 
5 5 2 1 0 1 2 5 2 1 1 0 1 2 5 5 2 1 2 1 1 
2 5 5 2 1 1 1 2 5 2 1 1 0 1 2 5 5 2 1 2 1 
1 2 5 5 2 0 1 1 2 5 2 1 0 0 1 2 5 5 1 1 2 
0 1 2 5 5 0 0 1 1 2 5 2 0 0 0 1 2 5 0 1 1 
0 0 1 2 5 0 0 0 1 1 2 5 0 0 0 0 1 2 0 0 1 
1 0 0 0 0 5 2 1 0 0 0 0 5 2 1 1 0 0 1 0 0 
2 1 0 0 0 5 5 2 1 0 0 0 2 5 2 1 1 0 2 1 0 
2 2 1 0 0 2 5 5 2 1 0 0 1 2 5 2 1 1 2 2 1 
2 2 2 1 0 1 2 5 5 2 1 0 1 1 2 5 2 1 2 2 2 
1 2 2 2 1 0 1 2 5 5 2 1 0 1 1 2 5 2 2 2 2 
0 1 2 2 2 0 0 1 2 5 5 2 0 0 1 1 2 5 1 2 2 
1 1 1 0 0 1 1 2 1 1 0 0 1 2 2 2 2 1 5 2 1 
1 1 1 1 0 0 1 1 2 1 1 0 0 1 2 2 2 2 2 5 2 
0 1 1 1 1 0 0 1 1 2 1 1 0 0 1 2 2 2 1 2 5
\end{verbatim}
\end{enumerate}

Les neuf instances $P1 \dots P9$ de ce jeu de données
sont décrites à partir des listes et matrices précédentes
ainsi:
\begin{enumerate}[label=(P\arabic*)]
\item $L1, M1$,
\item $L1, M2$,
\item $L2, M1$,
\item $L2, M2$,
\item $L3, M1$,
\item $L3, M2$,
\item $L4, M1$,
\item $L1, M3$,
\item $L5, M1$.
\end{enumerate}

\end{document}


