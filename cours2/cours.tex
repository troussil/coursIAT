\documentclass{beamer}
\mode<presentation>
\setbeamertemplate{navigation symbols}{}

% ----------------------------------------------------------------------
%footer
\setbeamertemplate{footline}
{
  \leavevmode
    \hfill
    \usebeamerfont{footer}
    ~~~~~~~~~~~~~~~~~~~~~~~~~~~~~~~~~~
    ~~~~~~~~~~~~~~~~~~~~~~~~~~~~~~~~~~
    \insertframenumber{} / \inserttotalframenumber
    ~~~~~
}

\usepackage[utf8]{inputenc}
\usepackage[T1]{fontenc}
\usepackage[french]{babel}

%figures
\usepackage{graphicx}
\graphicspath{{fig/}}
\DeclareGraphicsExtensions{.eps,.pdf,.jpg,.png}
\usepackage{tikz}

%math
\usepackage{amssymb}
\usepackage{amsmath}
\usepackage{cancel}

% ----------------------------------------------------------------------
%macros

\newcommand{\Z}{{\ensuremath\mathbb{Z}}}
\newcommand{\N}{{\ensuremath\mathbb{N}}}
\newcommand{\R}{{\ensuremath\mathbb{R}}}

\renewcommand{\vec}[1]{\mathbf{#1}}
\newcommand{\ve}[1]{\ensuremath{\vec{e}_{#1}}}

\DeclareMathOperator*{\argmin}{arg\,min}

% ----------------------------------------------------------------------
\title[]
 {4TC-IAT, Optimisation combinatoire}

\author[T. Roussillon]
 {Tristan Roussillon}

\date{2023}

\institute{INSA Lyon, TC}

\begin{document}

% ----------------------------------------------------------------------
\begin{frame}
  \titlepage
\end{frame}

% ----------------------------------------------------------------------
\section{\'Enoncé du problème}
% ----------------------------------------------------------------------

\begin{frame}
  \frametitle{Problème d'optimisation combinatoire}

  \[
  \text{(P)} \left\{
  \begin{array}{c}
    \min \ f(s) \ \text{tel que :} \\
    s \in S \, \text{et} \, A(s)
  \end{array}
  \right.
  \]

  \begin{itemize}
  \item $S$ est un ensemble discret fini de solutions,
  \item $f : S \mapsto \R$ est la fonction objectif,
  \item $A : S \mapsto \{\text{vrai},\text{faux}\}$ est le critère d'acceptation d'une solution. 
  \end{itemize}
\end{frame}

% ----------------------------------------------------------------------
\begin{frame}
  \frametitle{Ensemble des solutions acceptables}

  L'ensemble des solutions acceptables n'est généralement pas décrit par une liste exhaustive, car

  \begin{itemize}
  \item il peut-être difficile ou long de décider si une solution est acceptable ou non,
  \item et la liste pourrait être si longue qu'elle soit impossible à compléter en un temps raisonnable. 
  \end{itemize}
  
  Elle est plus souvent décrite implicitement par des propriétés.  

\end{frame}

% ----------------------------------------------------------------------
\begin{frame}
  \frametitle{De nombreux problèmes se ramènent à un problème
    d'optimisation combinatoire}

  \begin{itemize}
    \item \alert{problème d'affectation (assignment problem)}
    \item \alert{problème de coloration (graph coloring)}
    \item problème des surveillants de musée (art gallery problem)
    \item problème du sac à dos (knapsack problem)
    \item problème du voyageur de commerce (travelling salesman problem)
    \item problème de séquençage de tâches (job sequencing)
    \item \dots
  \end{itemize}
  
\end{frame}

% ----------------------------------------------------------------------
\begin{frame}
  \frametitle{Ex. 1: problème d'affectation}

  \begin{itemize}
  \item 4 ouvriers $\{1,2,3,4\}$ et 4 tâches $\{1,2,3,4\}$
  \item 1 matrice de coût
    $C := \left(
    \begin{array}{cccc}
      8 & 3 & 1 & 5 \\
      11 & 7 & 1 & 6 \\
      7 & 8 & 6 & 8 \\
      11 & 6 & 4 & 9 
    \end{array}
    \right)$ \\
    par ex. affecter l'ouvrier 4 à la tâche 3 coûte $c_{4,3} = 4$ 
  \item Trouver l'affectation de coût minimal \\
    par ex. l'identité a un coût de $8+7+6+9=30$. 
  \end{itemize}

\end{frame}

% ----------------------------------------------------------------------
\begin{frame}
  \frametitle{Ex. 1: problème d'affectation}

  \[
  \text{(P)} \left\{
  \begin{array}{c}
    \min \ f(s) \ \text{tel que :} \\
    s \in S \\
  \end{array}
  \right.
  \]

  \begin{itemize}
  \item $S$ est l'ensemble des permutations de $(1,2,3,4)$. 
  \item $f : S \mapsto \R$ est définie telle que, $\forall (t_1,t_2,t_3,t_4) \in S$, \\
    $f( (t_1,t_2,t_3,t_4) ) = \sum_{i=1}^4 c_{k,t_k}$. 
  \end{itemize}

  ~
  
  C'est un problème facile car
  \begin{itemize}
  \item toutes les solutions sont acceptables,
  \item il n'y a que $4! = 24$ solutions possibles. 
  \end{itemize}

\end{frame}

% ----------------------------------------------------------------------
\begin{frame}
  \frametitle{Ex. 1: problème d'affectation}

  S'il y a $n$ ouvriers et $n$ tâches,
  il y a $n!$ solutions possibles. 

  ~
  
  Par ex. si $n = 23$, $n! \approx 5 \cdot 10^{22}$, alors qu'une
  année ne compte qu'environ $3,16 \cdot 10^{13}$ microsecondes.

  ~
  
  Heureusement, il existe des algorithmes plus efficaces
  que la recherche exhaustive :
  \begin{itemize}
  \item algorithme hongrois en $O(n^3)$,
  \item algorithme des enchères,
  \item push-relabel, preflow-push, \dots
  \end{itemize}
  
\end{frame}

% ----------------------------------------------------------------------
\begin{frame}
  \frametitle{Ex. 2: coloration de graphe}

  \begin{figure}[htbp]
    \includegraphics<1>[page=1]{ex-graphe}%
    \includegraphics<2>[page=2]{ex-graphe}%
  \end{figure}

  
  \begin{itemize}
  \item Soit le graphe ci-dessus (graphe de Moser). 
  \item Trouver la coloration avec le plus petit nombre de couleurs. 
  \end{itemize}
  
\end{frame}

% ----------------------------------------------------------------------
\begin{frame}
  \frametitle{Ex. 2: coloration de graphe}
  
  \[
  \text{(P)} \left\{
  \begin{array}{c}
    \min \ f(s) \ \text{tel que :} \\
    s \in S \ \text{et} \ A(s) \\
  \end{array}
  \right.
  \]

  \begin{itemize}
  \item $S$ est l'ensemble des colorations du graphe, 
  \item $\forall s \in S$, $A(s)$ si pour toute arête,
    les couleurs des noeuds liés par l'arête sont différentes, 
  \item $f : S \mapsto \R$ est définie telle que, $\forall s \in S$, \\
    $f(s)$ est le nombre de couleurs présentes dans $s$.  
  \end{itemize}

  ~
  
  Il est très difficile d'énumérer l'ensemble des colorations acceptables. 
  
\end{frame}

% ----------------------------------------------------------------------
\begin{frame}
  \frametitle{Complexité du problème de coloration}

  On note $n$ le nombre de sommets. 
  
  Il y a des problèmes associé :
  \begin{itemize}
    \item décision:
      existe-t-il une coloration en $k$ couleurs ?
    \item recherche:
      exhiber une telle coloration. 
  \end{itemize}

  Le problème de décision est NP-complet pour $k>2$. 
      
%TODO  
%  décision, minimisation
%  NP
%  ...
  
\end{frame}


% ----------------------------------------------------------------------
\begin{frame}
  \frametitle{Cas d'école}

  
\end{frame}

% ----------------------------------------------------------------------
\begin{frame}<beamer>
  \frametitle{Sommaire}
  \tableofcontents[currentsection]
\end{frame}

% ----------------------------------------------------------------------
\section{Glouton}
% ----------------------------------------------------------------------

% ----------------------------------------------------------------------
\section{Backtracking}
% ----------------------------------------------------------------------

% ----------------------------------------------------------------------
\section{Branch and bound}
% ----------------------------------------------------------------------



%% % ----------------------------------------------------------------------
%% \begin{frame}
%%   \frametitle{Représentation arborescente et principe de séparation}
  
%%   \begin{itemize}
%%   \item $S \supseteq X$ est un ensemble de vecteurs possibles
%%     %$x^T = (x_1,\dots,x_n)$% et la racine d'un arbre. 
%%   \item On partitionne $S$ en sous-ensembles $S_{(v)} := \{ (x_1,\dots,x_n) \in S \ | \ x_1 = v \}$ \\
%%     %$\{ S_{(v)} \}_{v \in D(x_1)}$ 
%%     selon les valeurs prises par $x_1$ 
%%     %Ces sous-ensembles sont les fils de $S$ par relation d'inclusion.
%%   \item Pareil selon $x_2$ et ainsi de suite.
%%     %Chacun d'eux peut être partionné de la même façon selon $x_2$ et ainsi de suite.
%%   \end{itemize}

%%   {
%%     \centering
%%     \includegraphics<+>[width=0.9\textwidth,page=1]{arbre}
%%     \includegraphics<+>[width=0.9\textwidth,page=2]{arbre}
%%     \includegraphics<+>[width=0.9\textwidth,page=3]{arbre}
%%     \includegraphics<+>[width=0.9\textwidth,page=4]{arbre}
%%   }
  
%% \end{frame}

%% % ----------------------------------------------------------------------
%% \begin{frame}
%%   \frametitle{Principe d'évaluation}

%%   \begin{itemize}
%%   \item fonction d'évaluation $e$ : pour tout ensemble $T$ descendant de $S$,
%%     on choisit $e$ telle que $e(T) \leq \min_{x \in T} f(x)$
%%   \item si on connait une solution $\bar{x} \in S$ et qu'au noeud $T$ on a
%%     \[f(\bar{x}) < e(T) \leq \min_{x \in T} f(x) \]
%%   \item alors $T$ ne contient aucune solution optimale et ses descendants
%%     ne seront pas énumérés
%%   \end{itemize}
  
%% \end{frame}

%% % ----------------------------------------------------------------------
%% \begin{frame}
%%   \frametitle{Séparation-évaluation}

%%   On résoud (exactement) de nombreux problèmes en combinant ces deux principes
%%   (\emph{Branch and Bound}).

%%   Il y a d'innombrables variantes selon :
%%   \begin{itemize}
%%   \item la fonction d'évaluation
%%   \item le choix du noeud à explorer (et donc de la valeur à fixer)
%%   \item le choix de la variable à partir de laquelle on partitionne le noeud choisi
%%   \end{itemize}

%%   %% Note : on combine souvent ce principe avec la méthode des \emph{coupes intégrales}
%%   %% (\emph{Branch and Cut}). 
  
%% \end{frame}

%% % ----------------------------------------------------------------------
%% \begin{frame}
%%   \frametitle{Exemple d'un problème d'affectation}

%%   \only<1>{
%%   \begin{itemize}
%%   \item 4 ouvriers $\{1,2,3,4\}$ et 4 tâches $\{1,2,3,4\}$
%%   \item 1 matrice de coût
%%     $C := \left(
%%     \begin{array}{cccc}
%%       8 & 3 & 1 & 5 \\
%%       11 & 7 & 1 & 6 \\
%%       7 & 8 & 6 & 8 \\
%%       11 & 6 & 4 & 9 
%%     \end{array}
%%     \right)$ \\
%%     par ex. affecter l'ouvrier 4 à la tâche 3 coûte $c_{4,3} = 4$ 
%%   \item Trouver l'affectation de coût minimal
%%   \end{itemize}
%%   }

%%   \only<2>{
%%     \begin{itemize}
%%     \item On peut noter les inconnues et contraintes $x^T = (x_1,x_2,x_3,x_4)$,
%%       telles que $\forall i \in \{ 1,2,3,4\}, \ x_i \in \{ 1,2,3,4\}$ \\
%%       ($x_i$ contient le numéro de tâche affectée à l'ouvrier $i$).
      
%%     \item Pour exprimer la fonction objectif, on peut écrire pour tout $i$
%%       $x_i := \sum_{j=1}^4 j d_{i,j} = d_{i,1} + 2d_{i,2} + 3d_{i,3} + 4d_{i,4} $ \\
%%       ($d_{i,j} = 1$ si la tâche $j$ est affectée à l'ouvrier $i$, $d_{i,j} = 0$ sinon).
      
%%     \item Le problème s'exprime donc comme un (PNE)
%%       \[
%%       \left\{
%%       \begin{array}{c}
%%         \text{min} \ \sum_{i,j} c_{i,j}d_{i,j} \ \text{tel que :} \\
%%         \forall i \in \{ 1,2,3,4\}, \ x_i \in \{ 1,2,3,4\} \ \text{et} \ x_i = \sum_{j=1}^4 j d_{i,j} 
%%       \end{array}
%%       \right.
%%       \]
%%     \end{itemize}
%%   }
 
%% \end{frame}

%% % ----------------------------------------------------------------------
%% \begin{frame}
%%   \frametitle{Résolution par séparation-évaluation}

%%   \begin{itemize}
%%     \item ensemble des affectations possibles : $S$ ($|S| = 4! = 24$) 
%%     \item fonction d'évaluation $e$ : on somme les coûts minimaux des tâches non affectées, aux coûts des tâches déjà affectées
%%     \item choix de l'ouvrier selon lequel on sépare : arbitraire %(variable) 
%%     \item choix de la tâche à fixer en priorité : celle qui minimise $e$ %(valeur)
%%   \end{itemize}
%% \end{frame}

%% % ----------------------------------------------------------------------
%% \begin{frame}
%%   \frametitle{Déroulement de la méthode}

%%   \only<1>{
%%     \[
%%     \left(
%%     \begin{array}{cccc}
%%       8         & \alert{3} & \alert{1} & \alert{5} \\
%%       11        & 7         & 1         & 6 \\
%%       \alert{7} & 8         & 6         & 8 \\
%%       11        & 6         & 4         & 9 
%%     \end{array}
%%     \right)
%%     \]

%%     \[ e(S) = (7 + 3 + 1 + 5) = 16 \]

%%     \begin{itemize}
%%     \item $\Rightarrow 16$ minorant du coût minimum
%%     \item choix de l'ouvrier : $1$ 
%%     \item choix de la tâche à affecter\dots
%%     \end{itemize}
%%   }
%%   \only<2>{
%%     \begin{center}
%%     ouvrier 1 $\leftarrow$ tâche 1
%%     \end{center}
%%     \[
%%     \left(
%%     \begin{array}{cccc}
%%       \cancel{8}         & \cancel{3}  & \cancel{1}  & \cancel{5} \\
%%       \cancel{11}        & 7         & \alert{1} & \alert{6} \\
%%       \cancel{7}         & 8         & 6         & 8 \\
%%       \cancel{11}        & \alert{6} & 4         & 9 
%%     \end{array}
%%     \right)
%%     \]
%%   }
%%   \only<3,6>{
%%     \begin{center}
%%     \alert<6>{ouvrier 1 $\leftarrow$ tâche 2}
%%     \end{center}
%%     \[
%%     \left(
%%     \begin{array}{cccc}
%%       \cancel{8}& \cancel{3} & \cancel{1}  & \cancel{5} \\
%%       11             & \cancel{7} & \alert{1}        & \alert{6} \\
%%       \alert{7}      & \cancel{8} & 6                & 8 \\
%%       11             & \cancel{6} & 4                & 9 
%%     \end{array}
%%     \right)
%%     \]
%%   }
%%   \only<4>{
%%     \begin{center}
%%     ouvrier 1 $\leftarrow$ tâche 3
%%     \end{center}
%%     \[
%%     \left(
%%     \begin{array}{cccc}
%%       \cancel{8}& \cancel{3} & \cancel{1}  & \cancel{5} \\
%%       11             & 7               & \cancel{1}  & \alert{6} \\
%%       \alert{7}      & 8               & \cancel{6}  & 8 \\
%%       11             & \alert{6}       & \cancel{4}  & 9 
%%     \end{array}
%%     \right)
%%     \]
%%   }
%%   \only<5>{
%%     \begin{center}
%%     ouvrier 1 $\leftarrow$ tâche 4
%%     \end{center}
%%     \[
%%     \left(
%%     \begin{array}{cccc}
%%       \cancel{8}& \cancel{3} & \cancel{1}  & \cancel{5} \\
%%       11             & 7               & \alert{1}        & \cancel{6} \\
%%       \alert{7}      & 8               & 6                & \cancel{8} \\
%%       11             & \alert{6}       & 4                & \cancel{9} 
%%     \end{array}
%%     \right)
%%     \]
%%   }
%%   \begin{itemize}
%%     \item<2-> $e(S_{(1)}) = 8 + (6 + 1 + 6) = 21$
%%     \item<3-> \alert<6>{$e(S_{(2)}) = 3 + (7 + 1 + 6) = 17$}
%%     \item<4-> $e(S_{(3)}) = 1 + (7 + 6 + 6) = 20$
%%     \item<5-> $e(S_{(4)}) = 5 + (7 + 6 + 1) = 19$
%%   \end{itemize}
    
%% \end{frame}

%% % ----------------------------------------------------------------------
%% \begin{frame}
%%   \frametitle{Trace sous forme arborescente}

%%   {
%%     \includegraphics<+>[width=1\textwidth,page=1]{ex-bb}
%%     \includegraphics<+>[width=1\textwidth,page=2]{ex-bb}
%%   } 
%% \end{frame}

%% % ----------------------------------------------------------------------
%% \begin{frame}
%%   \frametitle{Résultat}


%%   \[
%%   e(S_{(4,3,1,2)}) = 5 + 1 + 7 + 6 = 19
%%   \]
  
%%   \[
%%   \left(
%%   \begin{array}{cccc}
%%     8 & 3 & 1 & \alert{5} \\
%%     11 & 7 & \alert{1} & 6 \\
%%     \alert{7} & 8 & 6 & 8 \\
%%     11 & \alert{6} & 4 & 9 
%%   \end{array}
%%   \right)
%%   \]

%%   \begin{block}{Minimum global}
%%   \begin{itemize}
%%     \item ouvrier 1 $\leftarrow$ tâche 4
%%     \item ouvrier 2 $\leftarrow$ tâche 3
%%     \item ouvrier 3 $\leftarrow$ tâche 1
%%     \item ouvrier 4 $\leftarrow$ tâche 2
%%   \end{itemize}
%%   \end{block}
%% \end{frame}

% ----------------------------------------------------------------------
\begin{frame}
  \frametitle{Méthodes de résolution}

  \begin{itemize}
  \item recherche exhaustive (très-très petits problèmes)
  \item \alert{\emph{Branch and Bound}} (séparation-évaluation)
  \item relaxation continue dans certains cas particuliers
  \item méthode des coupes (intégrales, mixtes)
  \item \emph{Branch and Bound} + coupes $=$ \emph{Branch and cut}
  \item programmation par contraintes
  \item méta-heuristiques : \\
      glouton, meilleur voisin, tabou, recuit-simulé, algorithme génétique, colonie de fourmis, \dots
  \end{itemize}
  
\end{frame}


% ----------------------------------------------------------------------
\section{Conclusion}
% ----------------------------------------------------------------------

% ----------------------------------------------------------------------
\begin{frame}<beamer>
  \frametitle{Sommaire}
  \tableofcontents[currentsection]
\end{frame}

% ----------------------------------------------------------------------
\begin{frame}
  \frametitle{Différentes classes de problèmes et d'algorithmes}

  \[
  \text{(P)} \left\{
  \begin{array}{c}
    \text{min} \ f(x) \ \text{tel que :} \\
    g_i(x) \leq 0, \ i \in I = \{1, \dots, m\} \\
    x \in S \subseteq \R^n
  \end{array}
  \right.
  \]

  \begin{itemize}
  \item Optimisation continue ($S = \R^n$)
    \begin{itemize}
    \item sans contrainte ($I = \emptyset$) \\
      \emph{descente de gradient}
    \item avec contraintes ($I \neq \emptyset$) 
      \begin{itemize}
        \item non-linéaires
        \item linéaires \\
        \emph{simplexe, points intérieurs}
      \end{itemize}
    \end{itemize}
  \item Optimisation combinatoire ($S = \Z^n$)\\
    \emph{branch and bound}, \emph{(meta)-heuristiques} 
  \end{itemize}
  
\end{frame}

% ----------------------------------------------------------------------
\begin{frame}
  \frametitle{Pour aller plus loin}

  \begin{itemize}
  \item Théorie de la dualité de Lagrange
  \end{itemize}


  \begin{itemize}
  \item Livre de référence sur les aspects théoriques de l'optimisation
  \end{itemize}

\begin{thebibliography}{alpha}
\bibitem{MM}
Michel Minoux, \emph{Programmation mathématique, Théorie et algorithmes}, Lavoisier, 2eme édition, 2008, 711 pp.
\end{thebibliography}

\end{frame}

\end{document}

