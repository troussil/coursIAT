\documentclass{beamer}
\mode<presentation>
\setbeamertemplate{navigation symbols}{}

% ----------------------------------------------------------------------
%footer
\setbeamertemplate{footline}
{
  \leavevmode
    \hfill
    \usebeamerfont{footer}
    ~~~~~~~~~~~~~~~~~~~~~~~~~~~~~~~~~~
    ~~~~~~~~~~~~~~~~~~~~~~~~~~~~~~~~~~
    \insertframenumber{} / \inserttotalframenumber
    ~~~~~
}

\usepackage[utf8]{inputenc}
\usepackage[T1]{fontenc}
\usepackage[french]{babel}

%figures
\usepackage{graphicx}
\graphicspath{{fig/}}
\DeclareGraphicsExtensions{.eps,.pdf,.jpg,.png}
\usepackage{tikz}

%alg
\usepackage[slide,linesnumbered,ruled,vlined]{algorithm2e}

%math
\usepackage{amssymb}
\usepackage{amsmath}
\usepackage{cancel}

% ----------------------------------------------------------------------
%macros

\newcommand{\Z}{{\ensuremath\mathbb{Z}}}
\newcommand{\N}{{\ensuremath\mathbb{N}}}
\newcommand{\R}{{\ensuremath\mathbb{R}}}

\renewcommand{\vec}[1]{\mathbf{#1}}
\newcommand{\ve}[1]{\ensuremath{\vec{e}_{#1}}}

\DeclareMathOperator*{\argmin}{arg\,min}

% ----------------------------------------------------------------------
\title[]
 {4TC-IAT, Optimisation combinatoire}

\author[T. Roussillon]
 {Tristan Roussillon}

\date{2023}

\institute{INSA Lyon, TC}

\begin{document}

% ----------------------------------------------------------------------
\begin{frame}
  \titlepage
\end{frame}

% ----------------------------------------------------------------------
\section{\'Enoncé du problème}
% ----------------------------------------------------------------------

\begin{frame}
  \frametitle{Problème d'optimisation combinatoire}

  \[
  \text{(P)} \left\{
  \begin{array}{c}
    \min \ f(s) \ \text{tel que :} \\
    s \in S \, \text{et} \, A(s)
  \end{array}
  \right.
  \]

  \begin{itemize}
  \item $S$ est un ensemble discret \emph{fini} de solutions,
  \item $f : S \mapsto \R$ est la fonction objectif,
  \item $A : S \mapsto \{\text{vrai},\text{faux}\}$ est le critère d'acceptation d'une solution. 
  \end{itemize}
\end{frame}

% ----------------------------------------------------------------------
\begin{frame}
  \frametitle{Ensemble des solutions acceptables}

  L'ensemble des solutions acceptables n'est généralement pas décrit par une liste exhaustive, car

  \begin{itemize}
  \item il peut être difficile ou long de décider si une solution est acceptable ou non,
  \item et la liste pourrait être si longue qu'elle soit impossible à compléter en un temps raisonnable. 
  \end{itemize}
  
  Elle est plus souvent décrite implicitement par des propriétés.  

\end{frame}

% ----------------------------------------------------------------------
\begin{frame}
  \frametitle{De nombreux problèmes se ramènent à un problème
    d'optimisation combinatoire}

  \begin{itemize}
    \item \alert{problème d'affectation (assignment problem)}
    \item \alert{problème de coloration (graph coloring)}
    \item problème des surveillants de musée (art gallery problem)
    \item problème du sac à dos (knapsack problem)
    \item problème du voyageur de commerce (travelling salesman problem)
    \item problème de séquençage de tâches (job sequencing)
    \item \dots
  \end{itemize}
  
\end{frame}

% ----------------------------------------------------------------------
\begin{frame}
  \frametitle{Ex. 1: problème d'affectation}

  \begin{itemize}
  \item 4 individus $\{1,2,3,4\}$ et 4 tâches $\{1,2,3,4\}$
  \item 1 matrice de coût
    $C := \left(
    \begin{array}{cccc}
      8 & 3 & 1 & 5 \\
      11 & 7 & 1 & 6 \\
      7 & 8 & 6 & 8 \\
      11 & 6 & 4 & 9 
    \end{array}
    \right)$ \\
    par ex. affecter l'individu 4 à la tâche 3 coûte $c_{4,3} = 4$ 
  \item Trouver l'affectation de coût minimal \\
    par ex. l'identité a un coût de $8+7+6+9=30$. 
  \end{itemize}

\end{frame}

% ----------------------------------------------------------------------
\begin{frame}
  \frametitle{Ex. 1: problème d'affectation}

  \[
  (Aff) \left\{
  \begin{array}{c}
    \min \ f(s) \ \text{tel que :} \\
    s \in S \\
  \end{array}
  \right.
  \]

  \begin{itemize}
  \item $S$ est l'ensemble des permutations de $(1,2,3,4)$. 
  \item $f : S \mapsto \R$ est définie telle que, $\forall s \in S$, \\
    $f(s) = \sum_{i=1}^4 c_{k,t_k}$, où $s = (t_1,t_2,t_3,t_4)$.  
  \end{itemize}

  ~
  
  C'est un problème facile car
  \begin{itemize}
  \item toutes les solutions sont acceptables,
  \item il n'y a que $4! = 24$ solutions possibles. 
  \end{itemize}

\end{frame}

% ----------------------------------------------------------------------
\begin{frame}
  \frametitle{Ex. 1: problème d'affectation}

  S'il y a $n$ individus et $n$ tâches,
  il y a $n!$ solutions possibles. 

  ~
  
  Par ex. si $n = 23$, $n! \approx 5 \cdot 10^{22}$, alors qu'une
  année ne compte qu'environ $3,16 \cdot 10^{13}$ microsecondes.

  ~
  
  Heureusement, il existe des algorithmes plus efficaces
  que la recherche exhaustive :
  \begin{itemize}
  \item algorithme hongrois en $O(n^3)$,
  \item algorithme des enchères,
  \item push-relabel, preflow-push, \dots
  \end{itemize}
  
\end{frame}

% ----------------------------------------------------------------------
\begin{frame}
  \frametitle{Ex. 2: coloration de graphe}

  \begin{figure}[htbp]
    \includegraphics<1>[page=1,width=0.7\textwidth]{ex-graphe}%
    \includegraphics<2>[page=2,width=0.7\textwidth]{ex-graphe}%
  \end{figure}

  
  \begin{itemize}
  \item Soit le graphe ci-dessus (graphe de Moser). 
  \item Trouver sa coloration avec le plus petit nombre de couleurs. 
  \end{itemize}
  
\end{frame}

% ----------------------------------------------------------------------
\begin{frame}
  \frametitle{Ex. 2: coloration de graphe}
  
  \[
  (Col) \left\{
  \begin{array}{c}
    \min \ f(s) \ \text{tel que :} \\
    s \in S \ \text{et} \ A(s) \\
  \end{array}
  \right.
  \]

  \begin{itemize}
  \item $S$ est l'ensemble des associations noeuds/couleurs, 
  \item $\forall s \in S$, $A(s)$ si pour toute arête,
    les couleurs, selon $s$, des noeuds reliés par l'arête sont différentes, 
  \item $f : S \mapsto \R$ est définie telle que, $\forall s \in S$, \\
    $f(s)$ est le nombre de couleurs présentes dans $s$.  
  \end{itemize}

  ~

  Une \emph{coloration} $s$ est une association noeuds/couleurs telle que $A(s)$.
  %(elle est acceptable).

  ~
  
  Il est difficile d'énumérer l'ensemble des colorations. 
  
\end{frame}

% ----------------------------------------------------------------------
\begin{frame}
  \frametitle{Complexité des problèmes de coloration de graphe}


  Il y a trois problèmes associés :
  \begin{itemize}
    \item décision:
      existe-t-il une coloration en $k$ couleurs ?
    \item recherche:
      exhiber une telle coloration.
    \item optimisation:
      trouver la coloration avec le plus petit $k$. %qui minimise le nombre de couleurs.
  \end{itemize}

  ~
  
  Pour $k \geq 3$,
  \begin{itemize}
  \item il n'y a pas d'algorithmes connus pour
    résoudre ces problèmes en temps polynomial.
  \item Le problème de décision est dans la classe NP
    %%si on nous donne une coloration à $k^$ couleurs
    %%on peut vérifier en temps polynomial $O(n^2)$ qu'elle est acceptable
    (NP-complet).
    %NP-complet = NP et NP-difficile
    %NP-difficile = au moins aussi difficile que tous ceux de la classe NP
  \end{itemize}
\end{frame}

% ----------------------------------------------------------------------
\begin{frame}
  \frametitle{Description des solutions}

  \[
  \text{(P)} \left\{
  \begin{array}{c}
    \min \ f(s) \ \text{tel que :} \\
    s \in S \, \text{et} \, A(s)
  \end{array}
  \right.
  \]

  \begin{block}{~}
  $S$ est généralement décrit par \emph{l'affectation} de \emph{valeurs}, parmi un ensemble de valeurs possibles, à chacune des \emph{variables} du problème. $A(S)$ ajoute des \emph{contraintes}.
  \end{block}
  
  \begin{block}{Ex. }
    \begin{itemize}
    \item (Aff): variables = individus, valeurs = tâches
    \item (Col): variables = noeuds, valeurs = couleurs
    \end{itemize}
  \end{block}
  
\end{frame}

% ----------------------------------------------------------------------
\begin{frame}<beamer>
  \frametitle{Sommaire : 4 méthodes de résolution}
  \tableofcontents[sections={2,3,5,6}]
\end{frame}

% ----------------------------------------------------------------------
\section{Algorithme glouton}
% ----------------------------------------------------------------------

% ----------------------------------------------------------------------
\begin{frame}
  \frametitle{Principe de l'algorithme glouton}

  Réaliser un choix localement optimal à chaque étape en espérant atteindre
  l'optimal global à la fin.

  ~
  
  La plupart du temps, on n'atteint pas l'optimum global, mais pour certains
  problèmes si, et pour d'autres, on peut garantir une certaine proximité. 

  % Chvatal : coloration de graphe sans P4
  
  % Claire Mathieu
  % Effectiveness of Local Search for Geometric Optimization
  % Local search yields approximation schemes for k-means and k-median in Euclidean and minor-free metrics

  ~
  
  L'algorithme est souvent facile à implémenter et s'exécute souvent rapidemment. 
  
\end{frame}

% ----------------------------------------------------------------------
\begin{frame}
  \frametitle{Résolution gloutonne du problème d'affectation}

  \[
  C := \left(
  \begin{array}{cccc}
    8 & 3 & 1 & 5 \\
    11 & 7 & 1 & 6 \\
    7 & 8 & 6 & 8 \\
    11 & 6 & 4 & 9 
  \end{array}
  \right)
  \]

  ~
  
  \begin{itemize}
  \item on traite les individus dans un certain ordre ; 
  \item pour chaque individu, on lui affecte la tâche encore disponible de moindre coût.
  \end{itemize}

  ~
  
  \begin{itemize}
  \item $c_{1,3} = 1$, $c_{2,4} = 6$, $c_{3,1} = 7$, $c_{4,2} = 6$.
  \item coût total : $1 + 6 + 7 + 6 = 20$. Optimal ?
  \end{itemize}
  
\end{frame}

% ----------------------------------------------------------------------
\begin{frame}
  \frametitle{Résolution gloutonne de la coloration}


  \begin{figure}[htbp]
    \includegraphics<1>[page=1,width=0.7\textwidth]{ex-graphe}%
    \includegraphics<2>[page=3,width=0.7\textwidth]{ex-graphe}%
    \includegraphics<3>[page=4,width=0.7\textwidth]{ex-graphe}%
    \includegraphics<4>[page=5,width=0.7\textwidth]{ex-graphe}%
    \includegraphics<5>[page=6,width=0.7\textwidth]{ex-graphe}%
    \includegraphics<6>[page=7,width=0.7\textwidth]{ex-graphe}%
    \includegraphics<7>[page=8,width=0.7\textwidth]{ex-graphe}%
    \includegraphics<8>[page=9,width=0.7\textwidth]{ex-graphe}%
    \includegraphics<9>[page=10,width=0.7\textwidth]{ex-graphe}%
  \end{figure}
  
  \begin{itemize}
  \item<2-> on traite les noeuds dans un certain ordre ; 
  \item<3-> pour chaque noeud, on lui affecte la couleur possible de plus petit indice. 
  \end{itemize}

  ~
  
  \begin{itemize}
  \item<9> $4$ couleurs au total. Optimal ?
  \end{itemize}

\end{frame}

% ----------------------------------------------------------------------
\begin{frame}
  \frametitle{Ordres possibles}

  \begin{itemize}
  \item random: noeuds triés aléatoirement. 
  \item largest first: noeuds triés par ordre de degré non croissant.
  \item smallest last: l'ordre $v_1,v_2,\dots,v_n$ est tel
    que $v_i$ a le plus petit degré dans le graphe
    ne contenant plus que les sommets $v_1,v_2,\dots,v_i$.
  \item DSATUR: l'ordre est construit en choisissant à chaque étape
    le noeud $v$ qui maximise la saturation (nombre de couleurs différentes
    déjà affectées aux voisins).
  \item \dots
  \end{itemize}
  
\end{frame}

% ----------------------------------------------------------------------
\begin{frame}
  \frametitle{Algorithme, complexité polynomiale}

 \begin{algorithm}[H]
  \caption{Glouton}
  \label{alg:glouton}
  \KwIn{probleme, description du probleme}
  \KwOut{affectation, tableau de valeurs}
  \tcp{pour toutes les variables, affecter}
  \ForEach{variable $i$}{
    affectation$[i]$ = valeurPossibleMin(probleme, affectation, i) \;
  }
  \Return affectation
 \end{algorithm}
  
\end{frame}

% ----------------------------------------------------------------------
\section{Backtracking}
% ----------------------------------------------------------------------

% ----------------------------------------------------------------------
\begin{frame}
  \frametitle{Backtracking}

  Méthode constructive pour résoudre un problème de décision :
  \begin{itemize}
  \item on traite les variables du problème dans un certain ordre ;
  \item pour l'affectation d'une valeur à une variable, on teste récursivement si une solution acceptable peut être construite à partir de cette affectation. Si ce n'est pas possible, on abandonne et on revient sur les affectations qui auraient été faites pour des variables précédentes (backtracking).
  \item on répond non si on a tout tenté sans succès, oui si on a trouvé une solution acceptable. 
  \end{itemize}
  
\end{frame}

% ----------------------------------------------------------------------
\begin{frame}
  \frametitle{Avertissement}

  On ne résoud pas un problème d'optimisation, mais le problème de \emph{décision}
  ou de \emph{recherche} associé.

  ~

  Dans le cas d'une fonction objectif discrète, le backtracking peut être
  utilisé pour voir si on peut faire mieux qu'une première approximation.

  ~
  
  Par ex. sur le problème de coloration précédent on a trouvé une solution
  à 4 couleurs avec l'algorithme glouton. Existe-t-il une coloration à 3 couleurs ?
  
\end{frame}

% ----------------------------------------------------------------------
\begin{frame}
  \frametitle{Coloration avec 3 couleurs ?}


  \begin{figure}[htbp]
    \includegraphics<1>[page=1,width=0.7\textwidth]{ex-graphe-backtracking}%
    \includegraphics<2>[page=2,width=0.7\textwidth]{ex-graphe-backtracking}%
    \includegraphics<3>[page=3,width=0.7\textwidth]{ex-graphe-backtracking}%
    \includegraphics<4>[page=4,width=0.7\textwidth]{ex-graphe-backtracking}%
    \includegraphics<5>[page=5,width=0.7\textwidth]{ex-graphe-backtracking}%
    \includegraphics<6>[page=6,width=0.7\textwidth]{ex-graphe-backtracking}%
    \includegraphics<7>[page=7,width=0.7\textwidth]{ex-graphe-backtracking}%
    \includegraphics<8>[page=8,width=0.7\textwidth]{ex-graphe-backtracking}%
    \includegraphics<9>[page=9,width=0.7\textwidth]{ex-graphe-backtracking}%
    \includegraphics<10>[page=10,width=0.7\textwidth]{ex-graphe-backtracking}%
  \end{figure}
  
  \begin{itemize}
  \item on traite les noeuds dans un certain ordre ; 
  \item pour chaque noeud, on teste les couleurs 1, 2 puis 3 ;
  \item on revient en arrière quand les contraintes sont violées.  
  \end{itemize}

  ~
  
  contrainte \only<1,2,4,5,7,10>{ OK !}\only<3,6,8,9>{\alert{ KO !}} \only<10>{ensuite ?}
  
\end{frame}

% ----------------------------------------------------------------------
\begin{frame}
  \frametitle{Algorithme, complexité exponentielle}

  \only<1>{
    \begin{algorithm}[H]
      \caption{backtracking(pb, sol, var, val)}
      \label{alg:backtracking}
      \If{estPossible(pb, sol, var, val)}{
        sol[var] $\leftarrow$ val \tcp*[l]{affectation}
        \lIf{estComplet(pb, sol)}{
          \Return (Vrai, sol)
        }
        \Else{
          varS $\leftarrow$ varSuivante(pb, var) \;
          val1 $\leftarrow$ premiereVal(pb, varS) \; 
          (flag, res) $\leftarrow$ backtracking(pb, sol, varS, val1) \;
          \lIf{flag}{
            \Return (flag, res)
          }
          \lElse(\tcp*[f]{annuler l'affectation}){
            sol[var] $\leftarrow \infty$ 
          }
        }
      }
      \tcp{poursuivre avec les valeurs suivantes \dots}
    \end{algorithm}
  }
  \only<2>{
    \setcounter{algocf}{1}
    \begin{algorithm}[H]
      \caption{backtracking(pb, sol, var, val)}
      \label{alg:backtracking}
      \setcounter{AlgoLine}{8}
      \tcp{\dots poursuivre avec les valeurs suivantes}
      valS $\leftarrow$ valSuivante(pb, val) \;
      \While{estAcceptable(pb, valS)}{
        (flag, res) $\leftarrow$ backtracking(pb, sol, var, valS) \;
        \lIf{flag}{
          \Return (flag, res)
        }
        \lElse(\tcp*[f]{continuer}){
          valS $\leftarrow$ valSuivante(pb, val)
        }
      }
      \Return (Faux, Rien)
    \end{algorithm}
  }
 
\end{frame}

% ----------------------------------------------------------------------
\begin{frame}
  \frametitle{Note sur l'implémentation}

  La compléxité en temps exponentielle est une limite intrinsèque de l'algorithme.

  ~
  
  Une autre limite est l'empilement des appels récursifs en mémoire.
  Elle va très vite être atteinte même pour des problèmes de taille raisonnable. 
  On peut la contourner en réécrivant l'algorithme sous forme itérative. 
  
\end{frame}

% ----------------------------------------------------------------------
\section{Intermède}
% ----------------------------------------------------------------------

% ----------------------------------------------------------------------
\begin{frame}
  \frametitle{Intermède}

  \only<1>{
    \begin{thebibliography}{alpha}
    \bibitem{AH}
      Alain Hertz, \emph{L'agrapheur, Intrigues policières à saveur mathématique},
      Presses internationales Polytechnique, 2010, 258 pp.
    \end{thebibliography}
  }

  \only<2>{
  \begin{figure}[htbp]
    \includegraphics[width=.5\textwidth]{sudoku}%
    \includegraphics[width=.5\textwidth]{graphe-sudoku}%
  \end{figure}
  
  Les cases A, B, C, F ne peuvent recevoir que 2, 4, ou 8;

  Les cases E et D, que 2 ou 8.  
  
  Peut-on affecter un seul chiffre aux cases A et C ?
  
  }
  
\end{frame}

% ----------------------------------------------------------------------
\section{Programmation dyamique}
% ----------------------------------------------------------------------

% ----------------------------------------------------------------------
\begin{frame}
  \frametitle{Programmation dynamique}

  Il s'agit de décomposer le problème en sous-problèmes, puis de résoudre les sous-problèmes,
  des plus petits aux plus grands en stockant les résultats intermédiaires.

  ~
  
  Cette approche a été théorisée et nommée ``programmation dynamique'' (pour faire bien)
  par Richard Bellman dans les années cinquantes. Elle repose sur la simple observation
  qu'un chemin optimal est formé de sous-chemins optimaux.
  %(ce qui est appelé maintenant principe d'optimalité de Bellman). 

  ~
  
\end{frame}
  
% ----------------------------------------------------------------------
\begin{frame}
  \frametitle{Retour sur le problème d'affectation}

  Il y a $4$ individus et $4$ tâches numérotés de $1$ à $4$.
  
  On note $l$ la liste des tâches déjà affectées aux $k$
  premiers individus, c-à-d. $k \in \{0,\dots,4\}$ est
  la longueur de $l$, et on définit :  

  \[
  (Aff_l) \left\{
  \begin{array}{c}
    \min \ f_l(s) \ \text{tel que :} \\
    s \in S_l \\
  \end{array}
  \right.
  \]

  \begin{itemize}
  \item $S_l$ est l'ensemble des arrangements des $4-k$ tâches
    non présentes dans $l$ parmi $4$.  
  \item si $l$ est complète ($k = 4$), on pose $\min_{s \in S_l} f_l(s) := 0$,
  \item sinon, $f_l : S_l \mapsto \R$ est le coût de l'affectation
      des $4-k$ tâches non présentes dans $l$ aux $k$ derniers individus.
  \end{itemize}

\end{frame}

% ----------------------------------------------------------------------
\begin{frame}
  \frametitle{Formule de récurrence}

  Pour $l$ et $|l| = k \in \{0,\dots,3\}$,  
  
  \[ \min_{s \in S_l} f_l(s) =
  \min_{t \in \{1,\dots,4\} \setminus l}
  \big( c_{(k+1),t} + \min_{s' \in S_{l \cup t}} f_{l \cup t}(s') \big), \]

  où $c_{(k+1),t}$ est le coût de l'affectation de la tâche $t$
  à l'individu $k+1$. 
  
\end{frame}

% ----------------------------------------------------------------------
\begin{frame}
  \frametitle{Premier niveau de récurrence}

  \only<1>{
    \begin{center}
    individu 1 $\leftarrow$ tâche 1
    \end{center}
    \[
    \left(
    \begin{array}{cccc}
      \cancel{8}         & \cancel{3}  & \cancel{1}  & \cancel{5} \\
      \cancel{11}        & 7         & \alert{1} & 6 \\
      \cancel{7}         & 8         & 6         & \alert{8} \\
      \cancel{11}        & \alert{6} & 4         & 9 
    \end{array}
    \right)
    \]
  }
  \only<2>{
    \begin{center}
    \alert<6>{individu 1 $\leftarrow$ tâche 2}
    \end{center}
    \[
    \left(
    \begin{array}{cccc}
      \cancel{8}& \cancel{3} & \cancel{1}  & \cancel{5} \\
      11             & \cancel{7} & \alert{1}        & 6 \\
      \alert{7}      & \cancel{8} & 6                & 8 \\
      11             & \cancel{6} & 4                & \alert{9} 
    \end{array}
    \right)
    \]
  }
  \only<3>{
    \begin{center}
    individu 1 $\leftarrow$ tâche 3
    \end{center}
    \[
    \left(
    \begin{array}{cccc}
      \cancel{8}& \cancel{3} & \cancel{1}  & \cancel{5} \\
      11             & 7               & \cancel{1}  & \alert{6} \\
      \alert{7}      & 8               & \cancel{6}  & 8 \\
      11             & \alert{6}       & \cancel{4}  & 9 
    \end{array}
    \right)
    \]
  }
  \only<4>{
    \begin{center}
    individu 1 $\leftarrow$ tâche 4
    \end{center}
    \[
    \left(
    \begin{array}{cccc}
      \cancel{8}& \cancel{3} & \cancel{1}  & \cancel{5} \\
      11             & 7               & \alert{1}        & \cancel{6} \\
      \alert{7}      & 8               & 6                & \cancel{8} \\
      11             & \alert{6}       & 4                & \cancel{9} 
    \end{array}
    \right)
    \]
  }
  \begin{itemize}
    \item<1-> $8 + (1 + 8 + 6) = 23$. 
    \item<2-> $3 + (1 + 7 + 9) = 20$. 
    \item<3-> $1 + (6 + 7 + 6) = 20$. 
    \item<4-> $5 + (1 + 7 + 6) = 19$. 
  \end{itemize}

\end{frame}

% ----------------------------------------------------------------------
\begin{frame}
  \frametitle{Algorithme, complexité exponentielle}

  \begin{algorithm}[H]
    \caption{minProgDyn(pb, sol)}
    \label{alg:minProgDyn}
    \If{estReduit(pb, sol)}{
      \Return (sol, coutSolution(pb, sol))
    }
    \Else{
      sols $\leftarrow$ liste vide\;
      \ForEach(\tcp*[f]{valeurs pour la 1ère variable}){val}{
        pb', sol' $\leftarrow$ reductionProbleme(pb, sol) \;
        solPart, coutPart $\leftarrow$ minProgDyn(pb', sol') \;
        solTot, coutTot $\leftarrow$ complete(pb, val, solPart, coutPart) \;
        ajouter (solTot, coutTot) à sols \; 
      }
      \Return minCout(sols)
    }
  \end{algorithm}
\end{frame}

% ----------------------------------------------------------------------
\begin{frame}
  \frametitle{Complexité de la programmation dynamique}

  En toute généralité, la complexité est exponentielle.
  Elle l'est par exemple pour le problème d'affectation.

  ~
  
  La complexité peut cependant être polynomiale quand le problème
  implique un ordre ou une hiérarchie naturelle:
  \begin{itemize}
  \item problème d'alignement de séquences,
  \item problème des pyramides de nombres, 
  \item problème de plus court chemin,
  \item problème du sac à dos,
  \item \dots
  \end{itemize}
  

\end{frame}

% ----------------------------------------------------------------------
\section{Branch and bound}
% ----------------------------------------------------------------------

% ----------------------------------------------------------------------
\begin{frame}
  \frametitle{Branch and Bound}

  Méthode de résolution de problèmes d'optimisation combinatoire
  introduite par Land et Doig en 1960 et nommée par Little en 1963.
%  pour le problème du voyageur de commerce. 

  ~
  
  Elle consiste
  \begin{itemize}
  \item à représenter l'ensemble des solutions acceptables par
    un arbre et à explorer certaines branches (\emph{branch}/\emph{separation})
  \item à borner inférieurement la fonction objectif
    pour toutes les solutions d'une branche (\emph{bound}/\emph{évaluation})
  \end{itemize}

  $\Rightarrow$ pas besoin d'explorer les branches dont la borne
  est supérieure à la meilleure solution courante !
  
\end{frame}

% ----------------------------------------------------------------------
\begin{frame}
  \frametitle{Représentation arborescente et principe de séparation}
  
  Pour une liste $l$ de $k$ valeurs, on note $S_l$ l'ensemble des
  solutions telles que les $k$ premières variables prennent les $k$
  premières valeurs de $l$ (et on pose $S := S_{\emptyset}$). 

  ~
  
  {
    \centering
    \includegraphics<+>[width=0.9\textwidth,page=1]{arbre}
    \includegraphics<+>[width=0.9\textwidth,page=2]{arbre}
    \includegraphics<+>[width=0.9\textwidth,page=3]{arbre}
    \includegraphics<+>[width=0.9\textwidth,page=4]{arbre}
  }
  
\end{frame}

% ----------------------------------------------------------------------
\begin{frame}
  \frametitle{Principe d'évaluation}

  \begin{itemize}
  \item fonction d'évaluation $e$ : pour tout ensemble $T$ descendant de $S$,
    on choisit $e$ telle que $e(T) \leq \min_{t \in T} f(t)$
  \item si on connait une solution $s \in S$ et qu'au noeud $T$ on a
    \[f(s) \leq e(T) \]
  \item alors $T$ ne contient aucune solution (strictement) meilleure que $s$
    et n'est donc pas exploré. 
  \end{itemize}
  
\end{frame}

% ----------------------------------------------------------------------
\begin{frame}
  \frametitle{Séparation-évaluation}

  On résoud de manière exacte de nombreux problèmes en combinant
  ces deux principes. C'est la méthode la plus courante pour
  résoudre un problème NP-complet. 

  ~
  
  La mise en oeuvre varie selon :
  \begin{itemize}
  \item la fonction d'évaluation. En l'absence de fonction d'évaluation
    ou de borne suffisamment haute, la méthode dégénère en recherche exhaustive.
    % $\rightsquigarrow$
  \item l'ordre des valeurs à affecter à une variable et l'ordre des variables.
    Il n'y a pas de bon choix a priori, cela dépend du problème.  
  \end{itemize}

  %% Note : on combine souvent ce principe avec la méthode des \emph{coupes intégrales}
  %% (\emph{Branch and Cut}). 
  
\end{frame}

% ----------------------------------------------------------------------
\begin{frame}
  \frametitle{Résolution du problème d'affectation}

  \begin{itemize}
    \item ensemble des affectations possibles : $S$ 
    \item fonction d'évaluation $e$ : on somme les coûts minimaux des tâches non affectées, aux coûts des tâches déjà affectées
    \item ordre des individus : arbitraire (dans l'ordre du problème) %(variable) 
    \item choix de la tâche à fixer en priorité : celle qui minimise $e$ %(valeur)
  \end{itemize}
\end{frame}

% ----------------------------------------------------------------------
\begin{frame}
  \frametitle{Déroulement de la méthode}

  \only<1>{
    \[
    \left(
    \begin{array}{cccc}
      8         & \alert{3} & \alert{1} & \alert{5} \\
      11        & 7         & 1         & 6 \\
      \alert{7} & 8         & 6         & 8 \\
      11        & 6         & 4         & 9 
    \end{array}
    \right)
    \]

    \[ e(S) = (7 + 3 + 1 + 5) = 16 \]

    \begin{itemize}
    \item $\Rightarrow 16$ minorant du coût minimum
    \item choix de l'individu : $1$ 
    \item choix de la tâche à affecter\dots
    \end{itemize}
  }
  \only<2>{
    \begin{center}
    individu 1 $\leftarrow$ tâche 1
    \end{center}
    \[
    \left(
    \begin{array}{cccc}
      \cancel{8}         & \cancel{3}  & \cancel{1}  & \cancel{5} \\
      \cancel{11}        & 7         & \alert{1} & \alert{6} \\
      \cancel{7}         & 8         & 6         & 8 \\
      \cancel{11}        & \alert{6} & 4         & 9 
    \end{array}
    \right)
    \]
  }
  \only<3,6>{
    \begin{center}
    \alert<6>{individu 1 $\leftarrow$ tâche 2}
    \end{center}
    \[
    \left(
    \begin{array}{cccc}
      \cancel{8}& \cancel{3} & \cancel{1}  & \cancel{5} \\
      11             & \cancel{7} & \alert{1}        & \alert{6} \\
      \alert{7}      & \cancel{8} & 6                & 8 \\
      11             & \cancel{6} & 4                & 9 
    \end{array}
    \right)
    \]
  }
  \only<4>{
    \begin{center}
    individu 1 $\leftarrow$ tâche 3
    \end{center}
    \[
    \left(
    \begin{array}{cccc}
      \cancel{8}& \cancel{3} & \cancel{1}  & \cancel{5} \\
      11             & 7               & \cancel{1}  & \alert{6} \\
      \alert{7}      & 8               & \cancel{6}  & 8 \\
      11             & \alert{6}       & \cancel{4}  & 9 
    \end{array}
    \right)
    \]
  }
  \only<5>{
    \begin{center}
    individu 1 $\leftarrow$ tâche 4
    \end{center}
    \[
    \left(
    \begin{array}{cccc}
      \cancel{8}& \cancel{3} & \cancel{1}  & \cancel{5} \\
      11             & 7               & \alert{1}        & \cancel{6} \\
      \alert{7}      & 8               & 6                & \cancel{8} \\
      11             & \alert{6}       & 4                & \cancel{9} 
    \end{array}
    \right)
    \]
  }
  \begin{itemize}
    \item<2-> $e(S_{(1)}) = 8 + (6 + 1 + 6) = 21$
    \item<3-> \alert<6>{$e(S_{(2)}) = 3 + (7 + 1 + 6) = 17$}
    \item<4-> $e(S_{(3)}) = 1 + (7 + 6 + 6) = 20$
    \item<5-> $e(S_{(4)}) = 5 + (7 + 6 + 1) = 19$
  \end{itemize}
    
\end{frame}

% ----------------------------------------------------------------------
\begin{frame}
  \frametitle{Trace sous forme arborescente}

  {
    \includegraphics<+>[width=1\textwidth,page=1]{ex-bb}
    \includegraphics<+>[width=1\textwidth,page=2]{ex-bb}
  } 
\end{frame}

% ----------------------------------------------------------------------
\begin{frame}
  \frametitle{Algorithme, complexité exponentielle}

  {\footnotesize%\small
    
  \begin{algorithm}[H]
    \caption{branchAndBound(pb)}
    \label{alg:brancheAndBound}
    minSol, minVal $\leftarrow$ glouton(pb) \tcp*[r]{meilleure solution courante}
    noeudsActifs $\leftarrow$ file vide \;
    r $\leftarrow$ racine(pb)~;~enfiler (r, evaluation(r)) à noeudsActifs \; 
    \While{noeudsActifs n'est pas vide}{
      (noeud, val) $\leftarrow$ extraire élément de noeudsActifs \;
      \If{estFeuille(noeud)}{
        \If{val < minVal}{
          minSol, minVal $\leftarrow$ solutionDecritePar(noeud), val \;
        }
      }
      \Else{
        enfants = separation(noeud) \tcp*[r]{branch}
        \ForEach{elem de enfants}{
          valElem $\leftarrow$ evaluation(elem) \tcp*[r]{bound} 
          \If{ valElem < minVal }{
            enfiler (elem, valElem) à noeudsActifs \;
          }
        }
      }
    }
    \Return minSol
  \end{algorithm}
  }
  
\end{frame}

% ----------------------------------------------------------------------
\begin{frame}
  \frametitle{Note sur l'implémentation}

  \begin{itemize}
  \item structure de données utilisée pour décrire un noeud : \\
    attention à ne pas recopier les mêmes données du problème.
  \item structure de données utilisée pour stocker les noeuds actifs :
    \begin{itemize}
    \item file (FIFO) : parcours en largeur de l'arbre
    \item pile (LIFO) : parcours en profondeur de l'arbre \\
      (recommandé quand on ne sait pas initialiser, ligne 1)
    \item file à priorité : explorer la branche la plus prometteuse \\
      (on peut s'arrêter à la première feuille trouvée, ligne 8) 
    \end{itemize}
  \end{itemize}
\end{frame}

% ----------------------------------------------------------------------
\section{Conclusion}
% ----------------------------------------------------------------------

% ----------------------------------------------------------------------
\begin{frame}
  \frametitle{Conclusion}

  Problème d'optimisation combinatoire
  
  ~
  
  \[
  \text{(P)} \left\{
  \begin{array}{c}
    \min \ f(s) \ \text{tel que :} \\
    s \in S \, \text{et} \, A(s)
  \end{array}
  \right.
  \]

  ~
  
  Le nombre de solutions acceptables est fini \dots

  mais peut être gigantesque !

  ~
  
  Certains problèmes sont dans P (problème d'affectation),
  
  d'autres sont NP-complets (problème de coloration).
  
\end{frame}

% ----------------------------------------------------------------------
\begin{frame}
  \frametitle{Méthodes de résolution (dans le cas NP-complet)}

  \begin{itemize}
  \item recherche exhaustive pour de très-très petits problèmes
  \item glouton : rapide mais ne donne en général pas le minimum
  \item (backtraking : test l'existence de solution pour un coût donné)
  \item programmation dynamique : retrouve le minimum mais peut dégénèrer en
    une recherche exhaustive selon le problème
  \item \alert{branch and bound} : retrouve le minimum mais peut dégénérer en
    une recherche exhaustive si l'évaluation est mauvaise
  \end{itemize}

\end{frame}

% ----------------------------------------------------------------------
\begin{frame}
  \frametitle{Autres méthodes de résolution}

  \begin{itemize}
  \item relaxation continue dans certains cas particuliers
  \item méthode des coupes (intégrales, mixtes)
  \item \emph{Branch and Bound} + coupes $=$ \emph{Branch and cut}
  \item programmation par contraintes
  \item méta-heuristiques : \\
    glouton, tabou, recuit-simulé, algorithme génétique, colonie de fourmis, \dots
  \item \dots
  \end{itemize}
  
\end{frame}

\end{document}

