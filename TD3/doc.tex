\documentclass[a4paper,francais]{article}
\usepackage[utf8]{inputenc}
\usepackage[T1]{fontenc}
\usepackage[french]{babel}

\usepackage{subfig}
\usepackage{graphicx}
\graphicspath{{fig/}}

\usepackage{amsmath}
\usepackage{amssymb}
\usepackage{amsthm}
\usepackage{cancel}
\usepackage{enumitem}

\usepackage{hyperref}

\usepackage{cprotect} %verbatim in footnote

\newcommand{\cad}{c.-à-d.}
\newcommand{\Z}{{\ensuremath\mathbb{Z}}}
\newcommand{\N}{{\ensuremath\mathbb{N}}}
\newcommand{\R}{{\ensuremath\mathbb{R}}}
\newtheorem{Theorem}{Theorem}
\newtheorem{Exemple}{Exemple}

%-------- enable or disable correction -----------------------------
\theoremstyle{definition}
\newtheorem{exercice}{Exercice}[section]
\newtheorem*{solution}{Solution}

\usepackage{comment}
\excludecomment{solution}% commenter/décommenter pour afficher/effacer l'impression des solutions


\title{Méthodes d'optimisation combinatoire appliquées au Channel Assignment Problem}
\author{Tristan Roussillon}
\date{6 mars 2023}

\begin{document}

\maketitle


\section{Définition du problème}
\label{sec:CAP}

Le problème d'allocation (ou assignation ou attribution) de fréquences consiste
à répartir un ensemble de fréquences sur les stations de base d'un réseau
tout en minimisant le nombre de fréquences utilisées. Il est habituellement
appelé \emph{Channel assignment problem} (CAP) ou encore \emph{minimum span frequency
  assignment problem} (MSFAP) en anglais, mais possède plusieurs variantes.
Le but de cette section est de poser une version de ce problème de manière rigoureuse
\footnote{Vous avez dans ce \href{http://www-sop.inria.fr/members/Frederic.Havet/Cours/channel.pdf}{document}
  dont nous reprenons plusieurs éléments, une description un peu plus approfondie
  des aspects de télécommunication mis en jeu.}.

Nous considérons un ensemble de noeuds $V$. Ils représentent des émetteurs.
Nous avons une fonction $p : V \rightarrow \N$ qui associe à chaque noeud, un nombre
entier appelé \emph{demande}. C'est le nombre de fréquences distinctes souhaité pour
un émetteur. Il y a un ordre total sur les fréquences de sorte qu'elles peuvent être
représentées par un ensemble d'entiers consécutifs $\{1, \dots, t\}$.
Nous avons de plus une fonction $l : V \times V \rightarrow \N$
qui associe à chaque paire de noeuds, un nombre entier qu'on peut appelé \emph{poids}
ou \emph{longueur}. 
Cette fonction décrit les contraintes d'interférence entre les émetteurs :
les fréquences utilisées par deux émetteurs $v_1$ et $v_2$ sont contraintes
d'être distantes de $l(v_1,v_2)$ ou plus afin d'éviter toute interférence.
Remarquons enfin que $V$ et $l$ définissent implicitement un graphe $G(V,E)$, 
où $E$ est défini comme l'ensemble des paires de poids non nul
$\{ (v_1,v_2) \in V \times V \ | \ l(v_1,v_2) > 0 \}$.  
 
Une \emph{solution} est une fonction $\phi : V \rightarrow \mathcal{P}(\{1,\dots,t\})$ telle que
le nombre d'éléments de $\phi(v)$ est égal à $p(v)$ pour chaque $v \in V$\footnote{La
  notation $\mathcal{P}$ désigne l'ensemble des parties d'un ensemble. Pour
  chaque $v \in V$, $\phi(v)$ est donc un sous-ensemble des fréquences $\{1,\dots,t\}$}.
Une solution est \emph{faisable} si pour toute paire $(v_1,v_2) \in V \times V$,
tout $f_1 \in \phi(v_1)$ et tout $f_2 \in \phi(v_2)$, on a $|f_1 - f_2| \geq l(v_1,v_2)$.
L'intervalle (\emph{span}) du problème, noté $\text{span}(V, p, l)$, est le plus petit $t$
tel qu'il existe une solution faisable.
\'Etant donnés $V, p, l$, le problème (CAP) consiste à déterminer $\text{span}(V, p, l)$
ainsi qu'une solution faisable $\phi$.

\begin{Exemple}
  \label{ex:triangle}
  $G$ est un cycle de 3 noeuds, qui ont chacun une demande de $1$.
  Les 3 arêtes sont de poids $3$. Le span est $7$.
\end{Exemple}

\begin{Exemple}
  \label{ex:carre}
  $G$ est un cycle de 4 noeuds, qui ont chacun une demande de $1$.
  Les 4 arêtes sont de poids $3$. Le span est $4$.
\end{Exemple}

\begin{Exemple}
  \label{ex:demande2}
  $G$ est un cycle de 5 noeuds avec en plus une boucle sur chaque noeud.
  Chaque noeud a une demande de $2$. Les boucles ont un poids de $2$,
  tandis que toutes les autres arêtes ont un poids de $1$. Le span est $5$. 
\end{Exemple}

\begin{exercice}
  Donnez une solution faisable $\phi$ réalisant le span donné
  dans les trois exemples précédents. 
\end{exercice}

\begin{exercice}
  Expliquez comment on peut transformer tout (CAP) faisant intervenir
  des demandes supérieures à $1$ en un (CAP) dont la demande est $1$
  pour chaque noeud. Illustrez avec l'exemple~\ref{ex:demande2}.
\end{exercice}

Grâce à cette transformation, nous pouvons maintenant considérer sans
perdre en généralité que la demande est $1$ pour chaque noeud.

\begin{exercice}
  La coloration de graphe consiste à trouver le plus petit nombre de
  couleurs permettant d'attribuer une couleur à chacun des noeuds
  d'un graphe, tout en garantissant que deux noeuds reliés par une
  arête sont de couleur différente.
  Expliquez sous quelles conditions (CAP) correspond à un problème
  de coloration de graphe. 
\end{exercice}

La coloration de graphe est un problème NP-complet. Or, (CAP) étant
au moins aussi difficile que la coloration de graphe d'après la
question précédente, on conclut que (CAP) est aussi un problème
NP-complet. 

\section{Résolution du problème}
\label{sec:resolution}

Même si (CAP) est NP-comlet, on peut résoudre certaines instances
de ce problème. Vous avez par exemple pu résoudre des instances de (CAP)
proposées dans les exemples \ref{ex:triangle}, \ref{ex:carre} et
\ref{ex:demande2}. 

\begin{exercice}
  Regardez et testez l'implémentation de l'algorithme glouton qui
  vous est fourni sur Moodle. Que pourrait-on améliorer sans changer
  l'implémentation de la fonction \verb!glouton! ? %ordre
  Pourquoi nous trouvons un span de 7 pour le problème correspondant
  à l'exemple~\ref{ex:demande2}, alors que nous savons qu'il est
  possible d'obtenir 5 ?
\end{exercice}

\begin{exercice}
  Complétez l'implémentation de la méthode \emph{Branch and Bound}
  à partir de l'algorithme donné en cours et en tenant compte des
  indications suivantes :
  \begin{itemize}
  \item ordre des stations : de $0$ à $n-1$, où $n$ est le nombre de stations. 
  \item séparation : pour une nouvelle station, on teste $m$ fréquences
    de $1$ à $m$, où $m$ est une borne supérieure sur le span (donné
    par l'algorithme glouton).
  \item évaluation : on se contente de la plus grande fréquence de
    la solution partielle.
  \item file : à priorité (voir la classe \verb!PriorityQueue! et
    ses méthodes \verb!enqueue! et \verb!dequeue!). 
  \end{itemize}
\end{exercice}



\end{document}


