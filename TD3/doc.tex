\documentclass[a4paper,francais]{article}
\usepackage[utf8]{inputenc}
\usepackage[T1]{fontenc}
\usepackage[french]{babel}

\usepackage{subfig}
\usepackage{graphicx}
\graphicspath{{fig/}}

\usepackage{amsmath}
\usepackage{amssymb}
\usepackage{amsthm}
\usepackage{cancel}
\usepackage{enumitem}

\usepackage{hyperref}

\usepackage{cprotect} %verbatim in footnote

\newcommand{\cad}{c.-à-d.}
\newcommand{\Z}{{\ensuremath\mathbb{Z}}}
\newcommand{\N}{{\ensuremath\mathbb{N}}}
\newcommand{\R}{{\ensuremath\mathbb{R}}}
\newtheorem{Theorem}{Theorem}
\newtheorem{Exemple}{Exemple}

%-------- enable or disable correction -----------------------------
\theoremstyle{definition}
\newtheorem{exercice}{Exercice}[section]
\newtheorem*{solution}{Solution}

\usepackage{comment}
\excludecomment{solution}% commenter/décommenter pour afficher/effacer l'impression des solutions

%alg
\usepackage[slide,linesnumbered,ruled,vlined]{algorithm2e}


\title{Méthode Branch and Bound appliquée au problème d'affectation}
\author{Tristan Roussillon}
\date{8 mars 2024}

\begin{document}

\maketitle

\section{Le problème et sa modélisation}

Le problème d'affectation a déjà été vu en cours. C'est un problème
d'optimisation combinatoire qu'on sait résoudre en temps polynomial.
Il a été choisi comme exemple pédagogique car il est facile à comprendre :
étant donnés $n$ individus et $n$ tâches numérotés de $0$ à $n-1$,
ainsi qu'une matrice de coût stockant pour chaque
couple $(i,j)$ le coût de l'affectation de la tâche $j$ à l'individu $i$,
il s'agit d'affecter chaque tâche à un individu distinct de façon à minimiser
le coût total des affectations.

Il est aussi facile de le modéliser. 
La matrice de coût \verb+C+ peut être représentée par une matrice,
un tableau à deux dimensions, une liste de liste selon le langage choisi,
pourvu que vous puissiez accéder facilement au coût d'une affectation,
par exemple en écrivant \verb+C[i][j]+. Une \emph{solution}, c'est-à-dire
l'affectation de toutes les tâches, ou une \emph{solution partielle},
c'est-à-dire l'affectation de certaines tâches, mais pas toutes, peut
être représentée de différentes manières. Il est suggéré d'utiliser
une liste de taille $m$, $0 \leq m \leq n$, contenant les
numéros des tâches déjà affectées aux $m$ premiers individus. Par
exemple, \verb+[3,0]+ signifie que la tâche 3 a été affectée à l'individu
0, que la tâche 0 a été affectée à l'individu 1, et que les individus
suivants n'ont pas encore de tâches. 
%% \begin{itemize}
%% \item par un tableau \verb+T+ de taille $n$, où le numéro de la tâche
%%   affectée au $i$-ème individu est donné par \verb+T[i]+ et avec la
%%   convention que \verb+T[i]+ contient une valeur spéciale qui ne se trouve
%%   pas entre $0$ et $n-1$ si le $i$-ème individu n'a pas encore de tâche.
%% \item par une liste de taille $m$, $0 \leq m \leq n$, contenant les tâches
%%   déjà affectées au $m$ premiers individus. 
%% \end{itemize}

\section{Implémentation de l'algorithme}

L'objectif du TD est d'implémenter la méthode \emph{Branch and Bound}
pour résoudre un problème d'affectation. 

\subsection{Rappel de la méthode}

La méthode \emph{Branch and Bound} a déjà été présentée en cours,
mais l'algorithme principal est précisé ici\footnote{Il est légèrement différent
que celui donné en cours mais le principe reste le même.}. 

  \begin{algorithm}[H]
    \caption{branchAndBound(C)}
    \label{alg:bb}
    minNoeud, minVal $\leftarrow$ meilleure solution courante et son évaluation \;
    r $\leftarrow$ racine de l'arbre \;
    noeudsActifs $\leftarrow$ pile vide \;
    ajouter (r, evaluation(r)) à noeudsActifs \; 
    \While{noeudsActifs n'est pas vide}{
      (noeud, val) $\leftarrow$ extraire élément de noeudsActifs \;
      \If{val < minVal}{
        \If{estFeuille(noeud)}{
          minNoeud, minVal $\leftarrow$ noeud, val \;
        }
        \Else{
          enfants $\leftarrow$ separation(noeud) \tcp*[r]{branch}
          \ForEach{elem de enfants}{
            valElem $\leftarrow$ evaluation(elem, C) \tcp*[r]{bound} 
            \If{ valElem < minVal }{
              ajouter (elem, valElem) à noeudsActifs \;
            }
          }
        }
      }
    }
    \Return minNoeud, minVal
  \end{algorithm}

\subsection{Structure de données}

L'algorithme est basé sur un arbre des solutions partielles. La racine
correspond à la solution partielle particulière dans laquelle aucune tâche
n'a été encore affectée. Une feuille correspond à une solution complète
dans laquelle toutes les tâches ont été affectées. 
Vous pouvez représenter un noeud de l'arbre par un objet qui contient
un champs représentant une solution partielle et un champs contenant
le nombre total d'individus $n$ pour en déduire quels sont les individus
ou quelles sont les tâches sans affectation.  

\subsection{Les fonctions auxiliaires (45 minutes max)}

L'algorithme appelle plusieurs fonctions\footnote{Je les écris sous la forme de fonctions, mais cela peut être des méthodes de l'objet représentant un noeud de l'arbre.} auxiliaires. Compte-tenu de ce qui a été dit plus haut, il est assez facile de créer la racine de l'arbre et de tester si un noeud donné est une feuille ou non. Les deux fonctions suivantes méritent des précisions :
\begin{itemize}
\item \verb+separation(noeud)+ retourne la liste des enfants du noeud donné. Ces enfants précisent la solution en ajoutant, à la solution partielle décrite par \verb+noeud+, l'affectation d'une tâche à un individu supplémentaire. Il faut donc être capable de parcourir toutes les tâches non encore affectées, pour, dans chaque enfant, en affecter une à un individu supplémentaire. 
\item \verb+evaluation(noeud, matrice)+ retourne une valeur qui est plus petite que le coût total de toutes les solutions qu'on peut obtenir à partir de la solution partielle décrite par \verb+noeud+. C'est la somme de deux termes. D'une part, la somme des coûts de toutes les tâches déjà affectées à un individu. D'autre part, la somme, pour toutes les tâches $j$ non encore affectées, des minima obtenus en trouvant l'individu $i$, parmi les individus qui n'ont pas encore de tâches, qui minimise \verb+C[i][j]+. Notez que dans le cas d'une feuille, c'est le coût total des affectations qui est retourné. Enfin, si vous n'avez pas compris comment calculer le deuxième terme, vous pouvez le mettre à zéro -- vous allez parcourir plus de noeuds, parce que votre fonction d'évaluation ne sera pas précise, mais vous pourrez avancer.    
\end{itemize}

\subsection{La fonction principale (45 minutes max)}

La fonction principale ne fait que traduire l'algorithme donné plus haut.
La pile des noeuds actifs peut être implémentée par une simple liste.
Pour l'initialisation de la meilleure solution courante en ligne 1,
une possibilité est de prendre l'identité,
c'est-à-dire affecter la tâche $i$ à l'individu $i$
pour tout $i=0,\dots,n-1$. 

\subsection{Exécution}

Voyez comment se déroule l'exécution pour la matrice de coût donné dans le cours
    \[ C := \left(
    \begin{array}{cccc}
      8 & 3 & 1 & 5 \\
      11 & 7 & 1 & 6 \\
      7 & 8 & 6 & 8 \\
      11 & 6 & 4 & 9 
    \end{array}
    \right) \]
ou d'autres matrices de votre choix.

Si vous avez fini, vous pouvez aussi tester l'impact du choix de la structure de données utilisée
pour stocker les noeuds actifs : pile, file, file à priorité.   

\end{document}


